\noindent Bài này bao gồm 3 phần độc lập. Trong tất cả các phần, hãy bỏ qua:
\begin{itemize}
  \item Sự tương tác của Mặt Trời với các thiên thể khác;
  \item Sự chuyển động của Trái Đất quanh Mặt Trời và tự quay quanh trục của nó;
  \item Lực cản của không khí.
\end{itemize}
Bạn có thể cần sử dụng một số thông số sau của Trái Đất:
\begin{itemize}
  \item Bán kính: $R=6,4.10^{6}\SI{}{\kilogram}$;
  \item Khối lượng: $M=6,0.10^{24}\SI{}{\kilogram}$;
  \item Gia tốc trọng trường trên bề mặt Trái Đất: $g=\SI{10}{\metre\second^{-2}}$.
\end{itemize}

\subsection*{Phần 1: Sự rơi của một tảng đá}


\noindent Một vật hình cầu bán kính $R=\SI{200}{\metre}$ ban đầu nằm cách bề mặt Trái Đất một khoảng $h=\SI{100}{\metre}$ và bắt đầu đầu rơi tự do không vận tốc đầu.

\begin{enumerate}
  \item Xác định thời gian $\tau$ kể từ lúc vật bắt đầu chuyển động cho đến khi nó chạm đất.
  \item Xác định vận tốc $v$ của vật tại thời điểm trước khi vật va chạm với Trái Đất.
\end{enumerate}
Xét hai trường hợp:
\begin{enumerate}
  \item[a.] Khối lượng riêng của vật bằng khối lượng riêng trung bình của Trái Đất.
  \item[b.] Khối lượng của vật bằng khối lượng Trái Đất (vật khi này là một ngôi sao neutron nhỏ).
\end{enumerate}

\subsection*{Phần 2: Tàu vũ trụ}
\noindent Một con tàu vũ trụ đang di chuyển trên quỹ đạo tròn quanh Trái Đất ở độ cao $h$ so với mặt đất. Giả sử độ cao $h$ và kích thước của tàu rất nhỏ so với bán kính Trái Đất. Xác định chu kì chuyển động $T$ của tàu. Xét hai trường hợp
\begin{enumerate}
  \item Khối lượng của tàu rất nhỏ so với khối lượng của Trái Đất.
  \item Khối lượng của tàu bằng khối lượng của Trái Đất.
\end{enumerate}

\subsection*{Phần 3: Chuẩn giờ}
\noindent Trên bề mặt Trái Đất, người ta xây một toà tháp có chiều cao lớn hơn bán kính Trái Đất một lượng nhỏ. Ở đỉnh tháp có gắn một con lắc đơn có chiều dài bằng bán kính Trái Đất. Ta có thể coi vật nặng như một chất điểm có khối lượng rất nhỏ so với khối lượng Trái Đất. Xác định chu kì dao động bé $T$ của con lắc này.