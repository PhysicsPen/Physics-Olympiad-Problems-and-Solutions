\documentclass[12pt]{article}
\usepackage[a4paper,top=2cm,bottom=2cm,left=2cm,right=2cm]{geometry}
\usepackage[utf8]{vietnam}
\usepackage{indentfirst}
\usepackage[hidelinks]{hyperref}
\usepackage{amsmath,amssymb,amsfonts}
\usepackage{wrapfig}
\usepackage{graphicx}
\usepackage{subcaption}
\usepackage{mwe}
\usepackage{siunitx}
\usepackage{enumitem}

%%%%%%%%%% PAGESTYLE %%%%%%%%%%
\usepackage{fancyhdr}
\pagestyle{fancy}
\fancyhf{}
\fancyhead[R]{\textbf{BelPhO 2024 - Chung kết Khối 11}}
\fancyhead[L]{\href{https://facebook.com/physicspen1111}{\textbf{Physics Pen}}}
\fancyfoot[C]{\thepage}

\begin{document}
%%%%%%%%%% FIRST PAGE STYLE %%%%%%%%%%
\thispagestyle{plain}

%%%%%%%%%% TITLE#1 %%%%%%%%%%
\begin{center}
  \fontfamily{cmss}\fontseries{b}\LARGE{\textbf{Đề thi và Lời giải\\ Olympic Vật lý Belarus năm 2024\\Vòng Chung kết Khối 11}}
\end{center}
\begin{center}
  \large\textit{\href{https://facebook.com/physicspen1111}{Sưu tầm và biên soạn bởi Physics Pen}}
\end{center}

\vspace{1cm}

%%%%%%%%%% PROBLEMS %%%%%%%%%%
\section*{Bài 1: Sự vĩ đại của khoa học}
\noindent Đĩa Faraday là một máy phát điện đơn giản, có cấu tạo bao gồm một đĩa kim loại làm bằng sắt (một loại vật liệu dẫn điện) có bán kính trong là $r_1$, bán kính ngoài là $r_2$ và độ dày là $h$. \\
\begin{figure}[H]
  \centering
  \includegraphics[width=0.5\textwidth]{Figures/Problems/Fig 1.1.png}
  \begin{center}
    \figurename{ 1}
  \end{center}
\end{figure}
\vspace{-0.5cm}
\noindent Đĩa được đặt trong một từ trường đều không đổi có cảm ứng từ $\vec{B}$ theo phương vuông góc với mặt phẳng của đĩa như được biểu diễn trong hình vẽ. Độ dẫn điện của sắt là $\sigma$.

\subsubsection*{Phần A: Mạch hở}
\noindent Trong phần này, người ta cho đĩa quay với tốc độ góc không đổi $\omega$ như Hình 1. Lúc này, đĩa chưa được nối với điện trở $R$ (không có đoạn mạch màu xanh dương).
\begin{enumerate}
  \item Xác định điện trường $\vec{E}$ bên trong đĩa và hiệu điện thế $V_0$ giữa mép trong và mép ngoài của đĩa. \textit{Gợi ý:} Trong một vật dẫn ở trạng thái cân bằng tĩnh điện, hợp lực tác dụng lên các điện tích tự do bằng không.
  \item Tìm điện trở $R_0$ của đĩa.
\end{enumerate}

\subsubsection*{Phần B: Phát điện}
\noindent Bây giờ, mép trong và mép ngoài của đĩa được nối với một điện trở $R$ như được minh họa bằng màu xanh dương trong hình vẽ. Các tiếp điểm không quay cùng với đĩa. Giả sử các tiếp điểm không có điện trở. Bỏ qua mọi ma sát.
\begin{enumerate}
  \item Xác định hiệu điện thế $V$ giữa mép trong và mép ngoài của đĩa. \textit{Gợi ý:} Lúc này, đĩa hoạt động như một nguồn điện không lý tưởng.
  \item Tính tổng công suất $P_0$ và công suất điện $P$ của nguồn.
  \item Tìm hiệu suất $\eta$ của máy phát.
  \item Tìm công suất điện cực đại $P_{\text{max}}$ và hiệu suất của nguồn khi hoạt động tại công suất đầu ra này.
\end{enumerate}

\subsubsection*{Phần C: Phanh tái sinh}
\noindent $N$ đĩa Faraday được dùng làm bánh xe và được kết nối cơ học với một đoàn tàu. Đoàn tàu có khối lượng $M = 400$ tấn và điện trở hiệu dụng $R$, mỗi bánh xe có khối lượng $m = 30$ kg. Phần duy nhất của tàu có chuyển động quay là các bánh xe. Sắt có độ dẫn điện $\sigma = 1{,}0 \cdot 10^7~\Omega^{-1} \cdot m^{-1}$, nhiệt dung riêng $c = 4{,}5 \cdot 10^2~J\,kg^{-1}\,K^{-1}$ và nhiệt độ nóng chảy $T = 1800~\text{K}$.
\begin{enumerate}
  \item Biểu diễn động năng của đoàn tàu dưới dạng $E = \frac{1}{2}I_{\text{eff}}\omega^2$ theo các đại lượng đã cho khi bánh xe quay với tốc độ góc $\omega$. Tìm $I_{\text{eff}}$. \textit{Gợi ý:} Moment quán tính của một vành tròn có khối lượng $M$, bán kính trong $R_1$ và bán kính ngoài $R_2$ đối với trục quay đi qua tâm và vuông góc với mặt phẳng vành là
        \begin{equation*}
          \frac{1}{2} M(R_2^2 + R_1^2)
        \end{equation*}
  \item Xác định động năng của đoàn tàu trong trường hợp $M \gg Nm$.
  \item Trong các ý còn lại của phần C, bạn có thể giả sử rằng $M \gg Nm$.
  \item Xác định tốc độ góc $\omega$ tại thời điểm $t$.
  \item Mất bao lâu để tốc độ của tàu giảm đi một nửa?
  \item Nếu tàu được hãm bằng cách nối tắt vành trong với vành ngoài của đĩa (lúc này $R = 0$), hãy ước lượng số lượng bánh xe tối thiểu để chúng không bị nóng chảy.
  \item Gia tốc khi hãm phanh có an toàn cho hành khách không? Lấy $B = 0{,}1\,$T, $r_1 = 10\,$cm, $r_2 = 50\,$cm và $h = 1{,}0\,$cm.
\end{enumerate}

\section*{Bài 2: Máy phát điện từ tính}
\begin{wrapfigure}{r}{9cm}
  \centering
  \includegraphics[width=0.5\textwidth]{images/Hinh 2.PNG}
  \vspace{-25px}
  \begin{center}
    \figurename{ 2}
  \end{center}
  \vspace{15px}
\end{wrapfigure}

\vspace{-30px}
\noindent Một ống thuỷ tinh mỏng, tiết diện đều, hai đầu bịt kín được uốn thành một nửa đường tròn bán kính $r$ (bán kính tiết diện của ống rất nhỏ so với $r$) sau đó được gắn cố định trên trên mặt sàn nằm ngang sao cho toàn bộ ống nằm trong mặt phẳng thẳng đứng như hình 2. Bên trong ống có một piston mỏng có khối lượng $m$ được làm bằng kim loại, diện tích của piston bằng với tiết diện của ống thuỷ tinh. Đường nối tâm đường tròn và vị trí của piston hợp với phương thẳng đứng một góc $\theta$. Hai bên piston đều chứa n mol khí lý tưởng, giả sử nhiệt độ của khí luôn bằng nhiệt độ $T$ của môi trường bên ngoài. Cho biết gia tốc trọng trường có độ lớn $g$, hằng số khí lý tưởng $R$, xem như tất cả các quá trình biến đổi trạng thái của khí đều chuẩn tĩnh. Bỏ qua mọi ma sát.\\
\vspace{-15pt}
\begin{enumerate}
  \item Khi nhiệt độ $T$ lớn hơn một nhiệt độ tới hạn $T_{C}$ nào đó, vị trí cân bằng bền của piston nằm ngay chính giữa ống ($\theta=0$). Hãy tìm biểu thức của $T_{C}$ và xác định tần số góc trong dao động bé của piston quanh vị trí cân bằng này.
  \item Khi $T=T_{C}$, hãy đánh giá tính ổn định của piston khi nó nằm cân bằng ở giữa ống ($\theta=0$).\\
\end{enumerate}
\vspace{-15px}
\noindent Đối với các câu hỏi bên dưới, xem $T_{C}$ như một thông số đã biết và không cần thay vào giá trị mà bạn đã tìm được ở ý 1.
\begin{enumerate}
  \setcounter{enumi}{2}
  \item Khi $T<T_{C}$, piston cân bằng tại vị trí góc $\theta_{0}$, tìm phương trình mà $\theta_{0}$ phải thoả mãn. Xác định biểu thức gần đúng khi nhiệt độ $T$ giảm nhẹ (xấp xỉ đến bậc thấp nhất khác 0).
  \item Khi $T<T_{C}$, xác định tần số góc $\omega_{0}$ trong dao động bé của piston quanh vị trí cân bằng tại $\theta_{0}$, tần số này bằng bao nhiêu khi nhiệt độ của khí lớn hơn và nhỏ hơn $T_{C}$ một chút.
  \item Khi $T<T_{C}$, giả sử vận tốc ban đầu của piston gần như bằng không, hãy tìm độ lớn vận tốc góc của piston khi nó di chuyển từ vị trí chính giữa $(\theta=0)$ đến vị trí góc lớn nhất $\theta$ mà nó có thể đi được.
\end{enumerate}

\section*{Bài 3: Sự đẳng thời và nguyên lý Fermat}
\begin{wrapfigure}[8]{l}{7cm}
  \centering
  \vspace{-4mm}
  \includegraphics[width=0.4\textwidth]{Figures/P3/Fig 3.1.png}
\end{wrapfigure}

\noindent Nguyên lý Fermat là cơ sở quan trọng của quang hình học, theo đó, đường truyền ánh sáng luôn là đường truyền sao cho thời gian truyền sáng là ngắn nhất. Nguyên lý này được Heron thành Alexandria phát biểu lần đầu vào thế kỷ I để giải thích hiện tượng phản xạ ánh sáng và được Pierre Fermat phát biểu vào năm 1662 dưới dạng một định luật tổng quát nhất của quang hình học.\\
\indent Nguyên lý Fermat từng được coi là một bí ẩn của khoa học: "Tại sao ánh sáng có thể tìm được đường đi nhanh nhất? Liệu ánh sáng có một bộ não nào đó?". Tất nhiên, ánh sáng không có não, nhưng hãy chứng minh rằng, bạn thì có!\\
\indent Trong bài này, bạn cần giải quyết một số bài toán quang học bằng cách sử dụng nguyên lý Fermat. Giả sử rằng bạn không biết các định luật về phản xạ và khúc xạ ánh sáng, nhưng bạn biết và tin vào nguyên lý Fermat. Bạn vẫn được phép sử dụng định luật truyền thẳng của ánh sáng. Trong các bài toán liên quan đến gương và thấu kính, hãy sử dụng phép xấp xỉ paraxial, tức hãy xem như các tia sáng di chuyển lân cận quang trục và tạo với quang trục các góc nhỏ.
\subsection*{Phần 1: Giới thiệu Toán học}
\noindent Để đơn giản hoá các thao tác tính toán, hãy sử dụng các công thức toán học sau:
\begin{enumerate}
  \item Chứng minh rằng khi $x\ll a$, ta sẽ có:
        \begin{equation}
          \label{eq:p31}
          \sqrt{a^{2}+x^{2}}\approx a+\frac{x^{2}}{2a}
        \end{equation}
  \item Một cung tròn nhỏ có thể xem gần đúng như một đoạn parabol. Xét một đường tròn bán kính $R$ có tâm nằm trên trục $y$ và đường tròn tiếp xúc với trục $x$. Chỉ ra rằng, phương trình của parabol tiếp xúc với đường tròn tại gốc toạ độ có dạng:
        \begin{equation}
          \label{eq:p32}
          y=\frac{x^{2}}{2R}
        \end{equation}
\end{enumerate}
Ngay cả khi bạn không thể chứng minh các công thức \eqref{eq:p31} và \eqref{eq:p32}, bạn vẫn có thể sử dụng chúng trong các phần tiếp theo.

\begin{figure}[h]
  \centering
  \includegraphics[width=0.3\textwidth]{Figures/P3/Fig 3.2.png}
\end{figure}

\subsection*{Phần 2: Sự đẳng thời}
\begin{wrapfigure}[8]{r}{8cm}
  \centering
  \vspace{-4mm}
  \includegraphics[width=0.45\textwidth]{Figures/P3/Fig 3.3.png}
\end{wrapfigure}

\noindent Một trường hợp cụ thể của nguyên lý Fermat là nguyên lý đẳng thời (hay nguyên lý Tautochronism). Nguyên lý này khẳng định rằng, với bất kì hệ quang học nào tạo ra ảnh, thời gian ánh sáng truyền từ nguồn điểm $A$ đến ảnh $A'$ của nó dọc theo bất kì con đường nào đều như nhau. Nói cánh khác, thời gian ánh sáng truyền từ nguồn $A$ đến ảnh $A'$ không phụ thuộc vào đường truyền ánh sáng.
\begin{enumerate}
  \item Cho một gương cầu lõm có bán kính $R$, tâm hình học $O$, quang tâm $C$ và quang trục $OC$ (Hình 3a). Bằng cách sử dụng nguyên lý đẳng thời, hãy chỉ ra rằng tất cả các tia $A'A$ song song với quang trục sau khi phản xạ trên gương sẽ giao nhau tại một điểm $F$ được gọi là tiêu điểm. Tìm tiêu cự $f=FC$ của gương.
  \item Cho một thấu kính phẳng lồi có bán kính cong của mặt lồi là $R$ và chiết suất của vật liệu làm thấu kính là $n$. $O$ là tâm hình học của mặt lồi và $C$ là quang tâm của thấu kính. $OC$ là quang trục của thấu kính (hình 3b). Bằng cách sử dụng nguyên lý đẳng thời, hãy chỉ ra rằng tất cả các tia $A'A$ song song với quang trục sau khi phản xạ trên gương sẽ giao nhau tại một điểm $F$ được gọi là tiêu điểm. Tìm tiêu cự $f=FC$ của thấu kính.
  \item Trên quang trục của một gương cầu lõm bán kính $R$ có đặt một nguồn sáng điểm $A$ ở khoảng cách $a=AC$ tính từ quang tâm của gương. Ảnh của điểm sáng này nằm ở khoảng cách $b=BC$ tính từ quang tâm của gương (hình 3c).
        \begin{enumerate}
          \item[a.] Sử dụng nguyên lý đẳng thời, chứng minh ảnh của điểm sáng này là ảnh thật.
          \item[b.] Thiết lập biểu thức biểu diễn mối quan hệ giữa $a,b$ và tiêu cự $f$ của gương. Biểu thức mà bạn tìm được gọi là "công thức gương cầu lõm"
        \end{enumerate}
  \item  Như bạn đã biết, các ảnh có thể là ảo (hình 3d).
        \begin{itemize}
          \item[a.] Cải tiến nguyên lý đẳng thời sao cho ta có thể dùng nó để xác định vị trí của các ảnh ảo.
          \item[b.] Chứng minh "công thức gương cầu lồi" có thể được viết dưới dạng tương tự "công thức gương cầu lõm" nếu bạn định nghĩa lại các đại lượng trong công thức đó.
          \item[c.] Giải thích nguyên lý đẳng thời bằng các lập luận trong phạm vi không quá 100 từ.
        \end{itemize}
\end{enumerate}


\begin{figure*}
  \centering
  \begin{subfigure}[b]{0.475\textwidth}
    \centering
    \includegraphics[width=0.7\textwidth]{Figures/P3/Fig 3.4.png}
    \caption{}
  \end{subfigure}
  \hfill
  \begin{subfigure}[b]{0.475\textwidth}
    \centering
    \includegraphics[width=0.85\textwidth]{Figures/P3/Fig 3.5.png}
    \caption{}
  \end{subfigure}
  \vskip\baselineskip
  \begin{subfigure}[b]{0.475\textwidth}
    \centering
    \includegraphics[width=0.8\textwidth]{Figures/P3/Fig 3.6.png}
    \caption{}
  \end{subfigure}
  \hfill
  \begin{subfigure}[b]{0.475\textwidth}
    \centering
    \includegraphics[width=0.8\textwidth]{Figures/P3/Fig 3.7.png}
    \caption{}
  \end{subfigure}
  \begin{center}
    \figurename{ 3}
  \end{center}
\end{figure*}

\subsection*{Phần 3: Nguyên lý Fermat}
\begin{enumerate}
  \item Điểm $A$ nằm trong môi trường có chiết suất $n_{1}$, điểm $B$ nằm trong môi trường có chiết suất $n_{2}$. Một tia sáng xuất phát từ $A$, sau khi khúc xạ qua mặt phân cách, sẽ đi đến $B$. Sử dụng nguyên lý Fermat, thiết lập công thức cho định luật khúc xạ ánh sáng.
  \item Cho một ống trụ rỗng có bán kính $R$, mặt trong của lớp trụ được mạ bạc nhờ đó ánh sáng có thể phản xạ trên nó. Khảo sát sự phản xạ của một tia sáng bên trong ống trụ trong mặt phẳng vuông góc với trục đối xứng của ống, chỉ ra tất cả các quỹ đạo khả dĩ của tia sáng $ACB$. Các điểm $A, B, C$ đều nằm trên mặt trong của ống trụ. Tâm của mặt cắt là $O$ và vị trí của điểm $C$ được xác định bởi góc $\phi$.
        \begin{enumerate}
          \item[a.] Xác định biểu thức của chiều dài quỹ đạo $ACB$ theo góc $\phi$ - $L(\phi)$. Phác hoạ đồ thị biểu diễn sự phụ thuộc này cho tất cả các giá trị khả thi của $\phi$.
          \item[b.] Xác định các giá trị của góc $\phi$ ứng với quỹ đạo thực của tia sáng. Chỉ ra các giá trị này trên đồ thị $L-\phi$.
        \end{enumerate}
\end{enumerate}

\begin{figure}[h]
  \centering
  \begin{subfigure}[b]{0.49\textwidth}
    \centering
    \includegraphics[width=0.7\textwidth]{Figures/P3/Fig 3.8.png}
  \end{subfigure}
  \hfill
  \begin{subfigure}[b]{0.49\textwidth}
    \centering
    \includegraphics[width=0.7\textwidth]{Figures/P3/Fig 3.9.png}
  \end{subfigure}
\end{figure}

\begin{enumerate}
  \setcounter{enumi}{2}
  \item Các kết luận:
        \begin{enumerate}
          \item[a.] Cải tiến công thức của nguyên lý Fermat để có thể mô tả tất cả các trường hợp đã khảo sát trong bài này.
          \item[b.] Lí giải bằng lời nguyên lý Fermat trong phạm vi không quá 100 từ.
        \end{enumerate}
\end{enumerate}

\newpage

%%%%%%%%%% TITLE#2 %%%%%%%%%%
\begin{center}
  \Large{\textbf{LỜI GIẢI THAM KHẢO}}
\end{center}

\vspace{5mm}

\section*{Bài 1: Sự vĩ đại của khoa học}
\noindent\textbf{1a.}
\begin{equation*}
  M=\frac{\pi r^{2}L}{2}(1+c)
\end{equation*}
\begin{equation*}
  d=\frac{\dfrac{\pi r^{2}L}{2}\left(\dfrac{4r}{3\pi}-\dfrac{4r}{3\pi}c\right)}{\dfrac{\pi r^{2}L}{2}(1+c)}=\frac{4r}{3\pi}\left(\frac{1-c}{1+c}\right)
\end{equation*}

\noindent\textbf{1b.} Momen quán tính của một khối trụ đặc có khối lượng $M$ và bán kính $r$ là $\dfrac{1}{2}Mr^2$ do đó
\begin{equation*}
  I=\frac{1}{2}\left(\frac{\pi r^{2}L}{2}\right)(1+c)r^{2}=\frac{\pi r^{4}L}{4}(1+c)
\end{equation*}

\noindent\textbf{2.} Định luật II Newton cho
\begin{equation*}
  \tau=I\ddot{\theta}
\end{equation*}
momen lực đối với trục đối xứng là
\begin{equation*}
  \begin{gathered}
    \tau=-Mgd\sin\theta \\
    I\ddot{\theta}=-Mgd\sin\theta \\
    \ddot{\theta}=-\frac{Mgd}{I}\sin\theta \\
    \Rightarrow T=\frac{2\pi}{\omega}=2\pi\sqrt{\frac{I}{Mgd}}
  \end{gathered}
\end{equation*}

\noindent\textbf{3a.} \\
\noindent\underline{\textbf{Cách 1}}: Phương trình chuyển động của khối trụ đối với trục quay đi qua điểm tiếp xúc
\begin{equation*}
  \tau=I_{\text{con}}\ddot{\theta}
\end{equation*}
momen lực tác dụng lên khối trụ tương tự như ý trên
\begin{equation*}
  \tau = -Mgd\sin\theta\approx - M g d \theta
\end{equation*}
momen quán tính đối với trục quay qua điểm tiếp xúc được cho bởi
\begin{equation*}
  I_{\text{con}}=I_{cm}+M(r-d)^{2}
\end{equation*}
\begin{equation*}
  I=I_{cm}+Md^{2}
\end{equation*}
\begin{equation*}
  \Rightarrow I_{\text{con}}=I+M((r-d)^{2}-d^{2})=I+M(r^{2}-2rd)
\end{equation*}
\begin{equation*}
  \Rightarrow\ddot{\theta}=-\frac{Mgd}{I_{con}}\theta
\end{equation*}
chu kì dao động
\begin{equation*}
  T=2\pi\sqrt{\frac{I_{con}}{Mgd}}=2\pi\sqrt{\frac{I+M(r^{2}-2rd)}{Mgd}}
\end{equation*}
có thể thấy, khi $d\rightarrow0$, chu kì $T\rightarrow0$.\\

\noindent\underline{\textbf{Cách 2}}: Hàm Lagrange là
\begin{equation*}
  L=T-U=\frac{1}{2}I_{con}\dot{\theta}^2-\frac{1}{2}Mgd\theta^2
\end{equation*}
phương trình chuyển động được cho bởi
\begin{equation*}
  \begin{gathered}
    \frac{d}{dt}\frac{\partial L}{\partial\theta}-\frac{\partial L}{\partial\theta}=0\Rightarrow I_{con}\ddot{\theta}+Mgd\theta=0 \\
    \Rightarrow\ddot{\theta}=-\frac{Mgd}{I_{con}}\theta                                                                           \\
    \Rightarrow T=2\pi\sqrt{\frac{I_{con}}{Mgd}}=2\pi\sqrt{\frac{I+M(r^{2}-2rd)}{Mgd}}
  \end{gathered}
\end{equation*}

\noindent\textbf{3b.} Vì khối trụ chỉ có thể lăn không trượt, năng lượng của nó là bảo toàn. Để thoát khỏi sự dao động, khối trụ phải có đủ động năng để khối tâm có thể lên đến vị trí thế năng cực đại
\begin{equation*}
  \frac{1}{2}Mv_{cm}^{2}+\frac{1}{2}I_{cm}\omega_{0}^{2}+Mg(r-d)=Mg(r+d)
\end{equation*}
\begin{equation*}
  \begin{gathered}
    \Rightarrow\frac{1}{2}M\omega_{0}^{2}(r-d)^{2}+\frac{1}{2}(I-Md^{2})\omega_{0}^{2}=2Mgd \\
    \Rightarrow\omega_{0}=\sqrt{\frac{4Mgd}{M(r-d)^{2}+(I-Md^{2})}}
  \end{gathered}
\end{equation*}
có thể thấy, khi $d\rightarrow0$, $\omega_0\rightarrow0$, điều này có nghĩa không có bất kì cân bằng bền nào trong giới hạn này và khối trụ sẽ tiếp tục di chuyển về một phía.\\
\newpage

\section*{Bài 2: Máy phát điện từ tính}
\vspace{-1cm}
\begin{wrapfigure}{r}{7cm}
  \centering
  \includegraphics[width=0.4\textwidth]{Figures/Fig 2S1.jpg}
\end{wrapfigure}

\noindent\textbf{1.} Vì $\sqrt{\sigma/(\rho g)}\ll S$, bề mặt của lớp thuỷ ngân dừng như là phẳng và bán kính của lớp thuỷ ngân lớn hơn rất nhiều so với độ dày của nó. Giả sử thuỷ ngân đã che phủ hoàn toàn đáy trên của hình trụ nhưng không tràn xuống mặt bên của nó. Xét một lớp chất lỏng có độ rồng $L$, cân bằng lực theo phương ngang cho:
\begin{equation*}
  F\cos\theta-F+p_{cp}Lh=\sigma L\cos\theta-\sigma L+p_{cp}Lh=0
\end{equation*}
trong đó $p_{cp}$ là áp suất thuỷ tĩnh trung bình theo độ dày của lớp thuỷ ngân:
\begin{equation*}
  p_{cp}=\frac{\rho gh}{2}
\end{equation*}
do đó:
\begin{equation*}
  \frac{\rho gh^{2}}{2}=\sigma(1-\cos\theta)\implies h=\sqrt{\frac{2\sigma(1-\cos\theta)}{\rho g}}
\end{equation*}
khi thuỷ ngân che phủ toàn bộ đáy trên của hình trụ, thể tích của nó bằng:
\begin{equation*}
  V_{0}=Sh=S\sqrt{\frac{2\sigma(1-\cos\theta)}{\rho g}}
\end{equation*}



\noindent\textbf{2.} Nếu đặt lên lớp thuỷ ngân một hình trụ có khối lượng $m$, áp suất thuỷ tĩnh tại mỗi điểm bên trong lớp thuỷ ngân sẽ tăng lên một lượng $mg/S_{k}$ với $S_{k}$ là diện tích tiếp xúc giữa thuỷ ngân và đáy dưới của hình trụ mà ta đặt lên. Áp suất thuỷ tĩnh trung bình trong lớp thuỷ ngân khi này:
\begin{equation*}
  p_{cp}=\frac{\rho gh}{2}+\frac{mg}{S_{k}}
\end{equation*}
điều kiện cân bằng:
\begin{equation*}
  \sigma(1-\cos\theta)=\frac{\rho gh^{2}}{2}+\frac{mgh}{S_{k}}
\end{equation*}
trong đó $h=V/S_{k}$ vì lớp thuỷ cân gần như là phẳng. Ta có:
\begin{equation*}
  \sigma(1-\cos\theta)=\frac{\rho gV^{2}}{2S_{k}^{2}}+\frac{mgV}{S_{k}^{2}}
\end{equation*}
thuỷ ngân sẽ lấp đầy hoàn toàn khe hở khi $S=S_{k}$, tức:
\begin{equation*}
  m_{1}=\frac{\sigma S^{2}(1-\cos\theta)}{gV}-\rho V^{2}
\end{equation*}

\begin{figure}[h]
  \centering
  \includegraphics[width=0.4\textwidth]{Figures/Fig 2S2.jpg}
\end{figure}

\noindent\textbf{3.} Trong quá trình thuỷ ngân tràn ra bên ngoài, tiếp tuyến của bề mặt thuỷ ngân sẽ quay $90^{\circ}$. Thuỷ ngân sẽ bắt đầu tràn ra khỏi khe hở khi áp suất thuỷ tĩnh vượt quá giá trị cho phép của lực căng bề mặt theo phương ngang. Xét hai trường hợp:
\begin{figure}[h]
  \centering
  \begin{subfigure}[b]{0.49\textwidth}
    \centering
    \includegraphics[width=0.8\textwidth]{Figures/Fig 2S3.jpg}
    \caption{Trường hợp 1}
  \end{subfigure}
  \hfill
  \begin{subfigure}[b]{0.49\textwidth}
    \centering
    \includegraphics[width=0.7\textwidth]{Figures/Fig 2S4.jpg}
    \caption{Trường hợp 2}
  \end{subfigure}
\end{figure}

\begin{enumerate}
  \item Trường hợp 1: $\theta<\dfrac{\pi}{2}$:\\
        Trong trường hợp này, tiếp tuyến của bề mặt thuỷ ngân sẽ không bao giờ nằm ngang, do đó, độ lớn lực căng bề mặt theo phương ngang sẽ đạt giá trị cực đại khi đường tiếp tuyến hợp một góc $\theta$ với mặt bên của hình trụ. Khi đó:
        \begin{equation*}
          F_{max}=\sigma L(1+\sin\theta)
        \end{equation*}
        như đã chỉ ra ở trên, điều kiện cân bằng là:
        \begin{equation*}
          \sigma(1+\sin\theta)=\frac{\rho gV^{2}}{2S^{2}}+\frac{m_{2}gV}{S^{2}}
        \end{equation*}
        suy ra:
        \begin{equation*}
          m_{2}=\frac{\sigma S^{2}(1+\sin\theta)}{gV}-\rho V^{2}
        \end{equation*}
  \item Trường hợp 2: $\theta\geqslant\dfrac{\pi}{2}$:
        Trong trường hợp này, tiếp tuyến của bề mặt thuỷ ngân có thể nằm theo phương ngang. Khi đó, độ lớn lực căng bề mặt theo phương ngang sẽ có giá trị cực đại:
        \begin{equation*}
          F_{max}=2\sigma L
        \end{equation*}
        lập luận tương tự trường hợp 1, ta được:
        \begin{equation*}
          m_{2}=\frac{2\sigma S^{2}}{gV}-\rho V^{2}
        \end{equation*}
\end{enumerate}




\setcounter{equation}{0}
\section*{Bài 3: Sự đẳng thời và nguyên lý Fermat}
\noindent\textbf{1.} Bên trong vật dẫn, điện trường bằng không. Các được sức phải vuông góc với mặt khoét. Do sự tích tụ các điện tích mặt, cường độ điện trường ở phía bên phải điện tích sẽ mạnh hơn phía bên trái do đó mật độ đường sức bên phải sẽ lớn hơn. Cuối cùng, lưu ý rằng điện tích là dương, các đường sức phải hướng ra ngoài điện tích. Với những lập luận trên, ta có thể phác hoạ các đường sức như hình 1.1:
\begin{figure}[h]
  \centering
  \begin{subfigure}[b]{0.49\textwidth}
    \centering
    \includegraphics[width=0.75\textwidth]{Figures/Solutions/Fig 3.1.png}
    \begin{center}
      \figurename{ 3.1}
    \end{center}
  \end{subfigure}
  \hfill
  \begin{subfigure}[b]{0.49\textwidth}
    \centering
    \includegraphics[width=0.85\textwidth]{Figures/Solutions/Fig 3.2.png}
    \begin{center}
      \figurename{ 3.2}
    \end{center}
  \end{subfigure}
\end{figure}

\noindent\textbf{2.} Để giải quyết bài toán này, ta sẽ sử dụng phương pháp ảnh điện với điện tích ảnh $q'<0$ được cách tâm hốc cầu một khoảng $d$ như hình 1.2. Điện thế tại một điểm $P$ nằm trên mặt hốc được cho bởi
\begin{equation*}
  \begin{gathered}
    \frac{1}{4\pi\varepsilon_{0}}\left(\frac{q}{\sqrt{z^{2}+R^{2}-2zR\cos\theta}}+\frac{q^{\prime}}{\sqrt{d^{2}+R^{2}-2dR\cos\theta}}\right)=\phi_{P}(\theta)=0 \\
    \Rightarrow\frac{q^{2}}{z^{2}+R^{2}-2zR\cos\theta}=\frac{{q^{\prime}}^{2}}{d^{2}+R^{2}-2dR\cos\theta} \\
    \Rightarrow q^{2}(d^{2}+R^{2})-{q^{\prime}}^{2}(z^{2}+R^{2})+2R({q^{\prime}}^{2}z-dq^{2})\cos\theta=0
  \end{gathered}
\end{equation*}
điều này phải đúng với mọi $\theta$
\begin{equation*}
  \begin{cases}
    q^{\prime2}z-dq^2=0                     \\
    q^2(d^2+R^2)-{q^{\prime2}}(z^2+R^2)=0 &
  \end{cases}
\end{equation*}
giải ra ta được
\begin{equation*}
  d=\frac{R^2}{z}\quad\text{và}\quad q^{\prime}=-\frac{qR}{z}
\end{equation*}
lực tác dụng lên điện tích điểm có độ lớn
\begin{equation*}
  |F|=\frac{1}{4\pi\varepsilon_{0}}\frac{|qq^{\prime}|}{(d-z)^{2}}=\frac{1}{4\pi\varepsilon_{0}}\frac{q^{2}Rz}{(R^{2}-z^{2})^{2}}
\end{equation*}

\noindent\textbf{3.} Theo định lý công - động năng, vận tốc tại $z=\dfrac{R}{2}$ được cho bởi
\begin{equation*}
  \frac{1}{2}m(v^{2}-0^{2})=\int_{0}^{k/2}F(z)\mathrm{d}z
\end{equation*}
\begin{equation*}
  \begin{gathered}
    v=\sqrt{\frac{2}{m}\int_{0}^{R/2}F(z)\mathrm{d}z}=\sqrt{\frac{q^{2}R}{2\pi\epsilon_{0}m}\int_{0}^{R/2}\frac{z}{(R^{2}-z^{2})^{2}}dz}=\sqrt{\frac{q^{2}}{2\pi\epsilon_{0}mR}\int_{0}^{1/2}\frac{u}{(1-u^{2})^{2}}du}\\
    v=\sqrt{\frac{1}{12\pi\varepsilon_{0}}\frac{q^{2}}{mR}}
  \end{gathered}
\end{equation*}

\end{document}