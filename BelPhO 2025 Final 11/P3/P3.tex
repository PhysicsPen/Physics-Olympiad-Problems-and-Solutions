\begin{wrapfigure}[8]{l}{4.95cm}
  \centering
  \vspace{-0.55cm}
  \includegraphics[width=0.275\textwidth]{Figures/Problems/Fig 3.1.png}
\end{wrapfigure}

\noindent Tính chất của các đồ uống có ga (bao gồm cả rượu Champagne) phần lớn được xác định bởi khí $CO_2$ hoà tan trong đó.\\
\indent Độ hoà tan của khí là lượng khí tối đa có thể hoà tan trong chất lỏng. Lượng khí (có thể được xác định bằng khối lượng, số mol, thể tích...) hoà tan trong một đơn vị thể tích chất lỏng được gọi là nồng độ khí hoà tan. Dung dịch chứa lượng khí tối đa có thể hoà tan được gọi là dung dịch bão hoà. Độ hoà tan có thể biểu diễn bằng nhiều đơn vị khác nhau như: g/l, mol/l, l/l,...\\
\indent Khi áp suất không quá cao, độ hoà tan của khí được mô tả bằng định luật Henry: độ hoà tan tỉ lệ với áp suất riêng phần của khí trên bề mặt chất lỏng:
\begin{equation*}
  C = kP
\end{equation*}
Trong đó, $C$ là nồng độ dung dịch bão hoà, $P$ là áp suất riêng phần của khí trên bề mặt chất lỏng, $k$ là hằng số Henry, phụ thuộc vào chất lỏng, khí và nhiệt độ.\\
\indent Bạn có thể cần tới:
\begin{itemize}
  \item Hằng số khí lý tưởng: $R = 8{,}31~\text{J}/(\text{mol}\cdot\text{K})$
  \item Quan hệ đơn vị áp suất: $1~\text{atm} = 1{,}01 \times 10^5~\text{Pa}$
  \item Chuyển đổi nhiệt độ: $T = t^\circ C + 273~\text{K}$
  \item Áp suất khí quyển: $P_0 = 1~\text{atm} = 1{,}01 \times 10^5~\text{Pa}$
  \item Khối lượng mol khí $CO_2$: $M = 44{,}0~\text{g/mol}$
\end{itemize}
Trong bài toán này, hãy sử dụng các giả thiết sau:
\begin{itemize}
  \item Khí $CO_2$ ở pha khí là khí lý tưởng;
  \item Thể tích dung dịch không phụ thuộc vào nồng độ khí hoà tan;
  \item Bỏ qua áp suất hơi bão hoà của nước.
\end{itemize}

\subsection*{Phần 1. Chuẩn bị}
\noindent Để đơn giản hóa quá trình tính toán, bạn nên sử dụng hệ đơn vị GLA, trong đó:
\begin{itemize}
  \item Đơn vị khối lượng: gram;
  \item Đơn vị thể tích: lít;
  \item Đơn vị áp suất: atm.
\end{itemize}
\begin{enumerate}
  \item Tìm giá trị số và đơn vị của hằng số khí lý tưởng $R$ trong hệ đơn vị GLA đã được giới thiệu.
\end{enumerate}
\noindent Ở nhiệt độ $t_0 = 0{,}00^\circ C$ và áp suất riêng phần $P_0 = 1{,}00~\text{atm}$, độ hoà tan của $CO_2$ trong nước là $k_v = 1{,}71~\text{lít khí}/\text{lít nước}$ (tức là 1 lít nước hoà tan được 1,71 lít khí ở nhiệt độ và áp suất đã cho). Trong các bước sau, ta sẽ biểu diễn nồng độ theo đơn vị gam/lít (khối lượng khí hoà tan trong 1 lít nước).
\begin{enumerate}
  \setcounter{enumi}{1}
  \item Xác định hệ số $k_m$ trong định luật Henry nếu nồng độ được tính bằng gam/lít và áp suất bằng atm. Tính giá trị số của hệ số này.
\end{enumerate}
\noindent Để sản xuất nước có gas mạnh, người ta bão hoà nước với khí $CO_2$ ở áp suất $P = 3{,}0~\text{atm}$ và nhiệt độ $t = 0{,}0^\circ C$.
\begin{enumerate}
  \setcounter{enumi}{2}
  \item Tính khối lượng khí $CO_2$ chứa trong chai 2 lít nước có gas mạnh ($V = 2{,}0~\text{lít}$).
\end{enumerate}
\noindent Sự phụ thuộc của hệ số Henry vào nhiệt độ được cho bởi:
\begin{equation*}
  k_m(t) = \frac{k_0}{1 + \alpha t}
\end{equation*}
trong đó, $k_0 = 3{,}4~\text{g}/(\text{l} \cdot \text{atm})$, $\alpha = 5{,}5 \times 10^{-2}~\text{K}^{-1}$, $t$ là nhiệt độ tính theo ${}^\circ$C.

\subsection*{Phần 2. Mở chai rượu}
\noindent Một chai rượu vang có thể tích $V_0 = 0{,}75~\text{lít}$ được đóng kín ở nhiệt độ $t_0 = 20^\circ C$. Trong đó còn $v = 0{,}15~\text{lít khí}$ (không có $CO_2$ hoà tan), phần còn lại chứa dung dịch rượu với khí hoà tan. Bỏ qua thể tích khí hoà tan trong pha khí. Sau quá trình lên men, mỗi chai sinh ra khoảng $m_0 = 7{,}5~\text{g}$ khí $CO_2$.
\begin{enumerate}
  \item Tìm sự phụ thuộc của áp suất trong chai vào nhiệt độ (trong khoảng $0^\circ C$ – $30^\circ C$). Vẽ đồ thị biểu diễn sự phụ thuộc đó.
\end{enumerate}
\noindent Một người trẻ thiếu kinh nghiệm mở chai ở nhiệt độ $t_0 = 25^\circ C$. Khi đó nút chai bị bật ra và rượu Champagne phun mạnh khỏi chai.
\begin{enumerate}
  \setcounter{enumi}{1}
  \item Ước lượng khối lượng rượu Champagne bị phun ra sau khi mở chai.
\end{enumerate}
Giả sử khi nút chai bị bật, áp suất khí trong chai giảm đột ngột về áp suất khí quyển, và khí $CO_2$ dư chuyển từ pha hoà tan sang pha khí cho đến khi đạt cân bằng bão hoà với dung dịch.
