\subsubsection*{Phần A: Khởi động}
\noindent \textbf{1.}
\begin{equation*}
  f = \frac{c}{\lambda}
\end{equation*}
Với $c$ là tốc độ ánh sáng trong chân không.\\

\noindent\textbf{2.}
\begin{equation*}
  E_0 = h f = \frac{hc}{\lambda}
\end{equation*}
Với $h$ là hằng số Planck.\\

\noindent\textbf{3.}
\begin{equation*}
  p_0 = \frac{E_0}{c} = \frac{h}{\lambda}
\end{equation*}

\textbf{4.} Sau thời gian $\Delta t$, có $n \Delta t$ photon được phát ra, mỗi photon có năng lượng $E_0$, do đó:
\begin{equation*}
  P = \frac{n E_0 \Delta t}{\Delta t} = n E_0
\end{equation*}
Suy ra:
\begin{equation*}
  n = \frac{P}{E_0} = \frac{\lambda P}{hc}
\end{equation*}

\subsubsection*{Phần B: Sóng hài bậc hai}
\noindent\textbf{1.} Bảo toàn năng lượng:
\begin{equation*}
  2 h f_1 = h f_2 \Rightarrow f_2 = 2 f_1
\end{equation*}
Suy ra:
\begin{equation*}
  \lambda_2 = \frac{1}{2} \lambda_1
\end{equation*}

\noindent\textbf{2.} Theo nguyên lí bất định Heisenberg:
\begin{equation*}
  \Delta t \Delta E = h \Delta t \Delta f_1 \geq h \Rightarrow \Delta f_1 \geq \frac{1}{\Delta t}
\end{equation*}
Do đó, độ rộng phổ nhỏ nhất là $1/\Delta t$.\\

\noindent\textbf{3.} Các tần số phân bố trên phạm vi $1/\Delta t$ quanh $f_1$, do đó:
\begin{equation*}
  f_1^{\pm} = f_1 \pm \frac{1}{2\Delta t}
\end{equation*}

\noindent\textbf{4.} Photon phát ra có tần số $f_2^{\pm}$ trong trường hợp các photon được hấp thụ có tần số $f_1^{\pm}$:
\begin{equation*}
  f_2^{\pm} = 2 f_1^{\pm}
\end{equation*}
Tần số nhỏ nhất và lớn nhất khi đó là:
\begin{equation*}
  f_2^{+/ -} = 2 f_1^{+/ -}
\end{equation*}
Độ rộng phổ:
\begin{equation*}
  \Delta f_2 = f_2^+ - f_2^- = 2(f_1^+ - f_1^-) = 2 \Delta f_1 = \frac{2}{\Delta t}
\end{equation*}

\noindent\textbf{5.} Ta có:
\begin{equation*}
  \Delta f_2 = \frac{1}{\Delta t_2} = \frac{2}{\Delta t} \Rightarrow \Delta t_2 = \frac{\Delta t}{2}
\end{equation*}
Thời lượng của xung mới có thể bị giảm một nửa, do đó, quá trình này mô tả một phương pháp tiềm năng để tạo ra các xung ngắn hơn.\\

\subsubsection*{Phần C: Bảo toàn động lượng}
\noindent\textbf{1.} Động lượng được cho bởi:
\begin{equation*}
  p = \frac{h}{\lambda}
\end{equation*}
Bước sóng $\lambda$ trong vật liệu được cho bởi:
\begin{equation*}
  \lambda = \frac{\lambda_0}{n}
\end{equation*}
Với $\lambda_0$ là bước sóng trong chân không. Do đó:
\begin{equation*}
  p = \frac{hn}{\lambda_0} = n p_0
\end{equation*}
Có thể thấy, chiết suất có vai trò quan trọng trong việc đảm bảo sự bảo toàn động lượng.\\

\noindent\textbf{2.} Định luật bảo toàn động lượng:
\begin{equation*}
  p_2 = 2p_1
\end{equation*}
Thừa số 2 có nghĩa là có 2 photon tần số $f_1$ đã bị hấp thụ. Ta có:
\begin{equation*}
  \frac{hf_2 n_2}{c} = \frac{2 h f_1 n_1}{c}
\end{equation*}
Sử dụng $f_2 = 2f_1$, ta được:
\begin{equation*}
  n_2 = n_1
\end{equation*}

\noindent\textbf{4.} Cường độ phụ thuộc vào độ dày của tinh thể. Nếu tinh thể đủ mỏng, các sóng được tạo ra tại mặt trước và mặt sau của tinh thể sẽ giao thoa gần như tăng cường với nhau. Khi tinh thể dày hơn, hai sóng này (từ phía trước và phía sau) sẽ giao thoa triệt tiêu, làm giảm cường độ của sóng phát ra.\\
