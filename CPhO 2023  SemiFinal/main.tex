\documentclass[12pt]{article}
\usepackage[a4paper,top=2cm,bottom=2cm,left=1.5cm,right=1.5cm]{geometry}
\usepackage[utf8]{vietnam}
\usepackage{graphicx}
\usepackage{wrapfig}
\usepackage{amsfonts,amsmath,amssymb}
\usepackage{siunitx}
\usepackage[hidelinks]{hyperref}
\usepackage{indentfirst}

%%%%%%%%%% EQUATION COUNTER & REFERENCE %%%%%%%%%% 
\usepackage{xassoccnt}
\newcounter{totalequations}
\DeclareAssociatedCounters{equation}{totalequations}
\let\theOldHequation\theHequation
\renewcommand{\theHequation}{\theOldHequation::\number\value{totalequations}}

%%%%%%%%%% PAGESTYLE %%%%%%%%%%
\usepackage{fancyhdr}
\pagestyle{fancy}
\fancyhf{}
\fancyhead[L]{\href{https://www.facebook.com/physicspen1111/}{\textbf{Physics Pen}}}
\fancyhead[R]{\textbf{CPhO 40 - Vòng Bán Kết}}
\fancyfoot[C]{\thepage}

\begin{document}

%%%%%%%%%% FIRST PAGE STYLE %%%%%%%%%%
\thispagestyle{empty}

%%%%%%%%%% TITLE %%%%%%%%%%
\begin{center}
  \fontfamily{cmss}\fontseries{b}\LARGE\textbf{
    Đề thi và Lời giải\\
    Olympic Vật lý Trung Quốc năm 2023\\
    Vòng Bán kết}
\end{center}

\begin{center}
  \href{https://www.facebook.com/physicspen1111/}{\large{\textit{Sưu tầm và biên soạn bởi Physics Pen}}}
\end{center}

\vspace{1cm}

%%%%%%%%%% PROBLEMS %%%%%%%%%%
\noindent\textbf{Câu I:}\\
\noindent Đĩa Faraday là một máy phát điện đơn giản, có cấu tạo bao gồm một đĩa kim loại làm bằng sắt (một loại vật liệu dẫn điện) có bán kính trong là $r_1$, bán kính ngoài là $r_2$ và độ dày là $h$. \\
\begin{figure}[H]
  \centering
  \includegraphics[width=0.5\textwidth]{Figures/Problems/Fig 1.1.png}
  \begin{center}
    \figurename{ 1}
  \end{center}
\end{figure}
\vspace{-0.5cm}
\noindent Đĩa được đặt trong một từ trường đều không đổi có cảm ứng từ $\vec{B}$ theo phương vuông góc với mặt phẳng của đĩa như được biểu diễn trong hình vẽ. Độ dẫn điện của sắt là $\sigma$.

\subsubsection*{Phần A: Mạch hở}
\noindent Trong phần này, người ta cho đĩa quay với tốc độ góc không đổi $\omega$ như Hình 1. Lúc này, đĩa chưa được nối với điện trở $R$ (không có đoạn mạch màu xanh dương).
\begin{enumerate}
  \item Xác định điện trường $\vec{E}$ bên trong đĩa và hiệu điện thế $V_0$ giữa mép trong và mép ngoài của đĩa. \textit{Gợi ý:} Trong một vật dẫn ở trạng thái cân bằng tĩnh điện, hợp lực tác dụng lên các điện tích tự do bằng không.
  \item Tìm điện trở $R_0$ của đĩa.
\end{enumerate}

\subsubsection*{Phần B: Phát điện}
\noindent Bây giờ, mép trong và mép ngoài của đĩa được nối với một điện trở $R$ như được minh họa bằng màu xanh dương trong hình vẽ. Các tiếp điểm không quay cùng với đĩa. Giả sử các tiếp điểm không có điện trở. Bỏ qua mọi ma sát.
\begin{enumerate}
  \item Xác định hiệu điện thế $V$ giữa mép trong và mép ngoài của đĩa. \textit{Gợi ý:} Lúc này, đĩa hoạt động như một nguồn điện không lý tưởng.
  \item Tính tổng công suất $P_0$ và công suất điện $P$ của nguồn.
  \item Tìm hiệu suất $\eta$ của máy phát.
  \item Tìm công suất điện cực đại $P_{\text{max}}$ và hiệu suất của nguồn khi hoạt động tại công suất đầu ra này.
\end{enumerate}

\subsubsection*{Phần C: Phanh tái sinh}
\noindent $N$ đĩa Faraday được dùng làm bánh xe và được kết nối cơ học với một đoàn tàu. Đoàn tàu có khối lượng $M = 400$ tấn và điện trở hiệu dụng $R$, mỗi bánh xe có khối lượng $m = 30$ kg. Phần duy nhất của tàu có chuyển động quay là các bánh xe. Sắt có độ dẫn điện $\sigma = 1{,}0 \cdot 10^7~\Omega^{-1} \cdot m^{-1}$, nhiệt dung riêng $c = 4{,}5 \cdot 10^2~J\,kg^{-1}\,K^{-1}$ và nhiệt độ nóng chảy $T = 1800~\text{K}$.
\begin{enumerate}
  \item Biểu diễn động năng của đoàn tàu dưới dạng $E = \frac{1}{2}I_{\text{eff}}\omega^2$ theo các đại lượng đã cho khi bánh xe quay với tốc độ góc $\omega$. Tìm $I_{\text{eff}}$. \textit{Gợi ý:} Moment quán tính của một vành tròn có khối lượng $M$, bán kính trong $R_1$ và bán kính ngoài $R_2$ đối với trục quay đi qua tâm và vuông góc với mặt phẳng vành là
        \begin{equation*}
          \frac{1}{2} M(R_2^2 + R_1^2)
        \end{equation*}
  \item Xác định động năng của đoàn tàu trong trường hợp $M \gg Nm$.
  \item Trong các ý còn lại của phần C, bạn có thể giả sử rằng $M \gg Nm$.
  \item Xác định tốc độ góc $\omega$ tại thời điểm $t$.
  \item Mất bao lâu để tốc độ của tàu giảm đi một nửa?
  \item Nếu tàu được hãm bằng cách nối tắt vành trong với vành ngoài của đĩa (lúc này $R = 0$), hãy ước lượng số lượng bánh xe tối thiểu để chúng không bị nóng chảy.
  \item Gia tốc khi hãm phanh có an toàn cho hành khách không? Lấy $B = 0{,}1\,$T, $r_1 = 10\,$cm, $r_2 = 50\,$cm và $h = 1{,}0\,$cm.
\end{enumerate}

\noindent\textbf{Câu II:}\\
\begin{wrapfigure}{r}{9cm}
  \centering
  \includegraphics[width=0.5\textwidth]{images/Hinh 2.PNG}
  \vspace{-25px}
  \begin{center}
    \figurename{ 2}
  \end{center}
  \vspace{15px}
\end{wrapfigure}

\vspace{-30px}
\noindent Một ống thuỷ tinh mỏng, tiết diện đều, hai đầu bịt kín được uốn thành một nửa đường tròn bán kính $r$ (bán kính tiết diện của ống rất nhỏ so với $r$) sau đó được gắn cố định trên trên mặt sàn nằm ngang sao cho toàn bộ ống nằm trong mặt phẳng thẳng đứng như hình 2. Bên trong ống có một piston mỏng có khối lượng $m$ được làm bằng kim loại, diện tích của piston bằng với tiết diện của ống thuỷ tinh. Đường nối tâm đường tròn và vị trí của piston hợp với phương thẳng đứng một góc $\theta$. Hai bên piston đều chứa n mol khí lý tưởng, giả sử nhiệt độ của khí luôn bằng nhiệt độ $T$ của môi trường bên ngoài. Cho biết gia tốc trọng trường có độ lớn $g$, hằng số khí lý tưởng $R$, xem như tất cả các quá trình biến đổi trạng thái của khí đều chuẩn tĩnh. Bỏ qua mọi ma sát.\\
\vspace{-15pt}
\begin{enumerate}
  \item Khi nhiệt độ $T$ lớn hơn một nhiệt độ tới hạn $T_{C}$ nào đó, vị trí cân bằng bền của piston nằm ngay chính giữa ống ($\theta=0$). Hãy tìm biểu thức của $T_{C}$ và xác định tần số góc trong dao động bé của piston quanh vị trí cân bằng này.
  \item Khi $T=T_{C}$, hãy đánh giá tính ổn định của piston khi nó nằm cân bằng ở giữa ống ($\theta=0$).\\
\end{enumerate}
\vspace{-15px}
\noindent Đối với các câu hỏi bên dưới, xem $T_{C}$ như một thông số đã biết và không cần thay vào giá trị mà bạn đã tìm được ở ý 1.
\begin{enumerate}
  \setcounter{enumi}{2}
  \item Khi $T<T_{C}$, piston cân bằng tại vị trí góc $\theta_{0}$, tìm phương trình mà $\theta_{0}$ phải thoả mãn. Xác định biểu thức gần đúng khi nhiệt độ $T$ giảm nhẹ (xấp xỉ đến bậc thấp nhất khác 0).
  \item Khi $T<T_{C}$, xác định tần số góc $\omega_{0}$ trong dao động bé của piston quanh vị trí cân bằng tại $\theta_{0}$, tần số này bằng bao nhiêu khi nhiệt độ của khí lớn hơn và nhỏ hơn $T_{C}$ một chút.
  \item Khi $T<T_{C}$, giả sử vận tốc ban đầu của piston gần như bằng không, hãy tìm độ lớn vận tốc góc của piston khi nó di chuyển từ vị trí chính giữa $(\theta=0)$ đến vị trí góc lớn nhất $\theta$ mà nó có thể đi được.
\end{enumerate}

\noindent\textbf{Câu III:}\\
\begin{wrapfigure}[8]{l}{7cm}
  \centering
  \vspace{-4mm}
  \includegraphics[width=0.4\textwidth]{Figures/P3/Fig 3.1.png}
\end{wrapfigure}

\noindent Nguyên lý Fermat là cơ sở quan trọng của quang hình học, theo đó, đường truyền ánh sáng luôn là đường truyền sao cho thời gian truyền sáng là ngắn nhất. Nguyên lý này được Heron thành Alexandria phát biểu lần đầu vào thế kỷ I để giải thích hiện tượng phản xạ ánh sáng và được Pierre Fermat phát biểu vào năm 1662 dưới dạng một định luật tổng quát nhất của quang hình học.\\
\indent Nguyên lý Fermat từng được coi là một bí ẩn của khoa học: "Tại sao ánh sáng có thể tìm được đường đi nhanh nhất? Liệu ánh sáng có một bộ não nào đó?". Tất nhiên, ánh sáng không có não, nhưng hãy chứng minh rằng, bạn thì có!\\
\indent Trong bài này, bạn cần giải quyết một số bài toán quang học bằng cách sử dụng nguyên lý Fermat. Giả sử rằng bạn không biết các định luật về phản xạ và khúc xạ ánh sáng, nhưng bạn biết và tin vào nguyên lý Fermat. Bạn vẫn được phép sử dụng định luật truyền thẳng của ánh sáng. Trong các bài toán liên quan đến gương và thấu kính, hãy sử dụng phép xấp xỉ paraxial, tức hãy xem như các tia sáng di chuyển lân cận quang trục và tạo với quang trục các góc nhỏ.
\subsection*{Phần 1: Giới thiệu Toán học}
\noindent Để đơn giản hoá các thao tác tính toán, hãy sử dụng các công thức toán học sau:
\begin{enumerate}
  \item Chứng minh rằng khi $x\ll a$, ta sẽ có:
        \begin{equation}
          \label{eq:p31}
          \sqrt{a^{2}+x^{2}}\approx a+\frac{x^{2}}{2a}
        \end{equation}
  \item Một cung tròn nhỏ có thể xem gần đúng như một đoạn parabol. Xét một đường tròn bán kính $R$ có tâm nằm trên trục $y$ và đường tròn tiếp xúc với trục $x$. Chỉ ra rằng, phương trình của parabol tiếp xúc với đường tròn tại gốc toạ độ có dạng:
        \begin{equation}
          \label{eq:p32}
          y=\frac{x^{2}}{2R}
        \end{equation}
\end{enumerate}
Ngay cả khi bạn không thể chứng minh các công thức \eqref{eq:p31} và \eqref{eq:p32}, bạn vẫn có thể sử dụng chúng trong các phần tiếp theo.

\begin{figure}[h]
  \centering
  \includegraphics[width=0.3\textwidth]{Figures/P3/Fig 3.2.png}
\end{figure}

\subsection*{Phần 2: Sự đẳng thời}
\begin{wrapfigure}[8]{r}{8cm}
  \centering
  \vspace{-4mm}
  \includegraphics[width=0.45\textwidth]{Figures/P3/Fig 3.3.png}
\end{wrapfigure}

\noindent Một trường hợp cụ thể của nguyên lý Fermat là nguyên lý đẳng thời (hay nguyên lý Tautochronism). Nguyên lý này khẳng định rằng, với bất kì hệ quang học nào tạo ra ảnh, thời gian ánh sáng truyền từ nguồn điểm $A$ đến ảnh $A'$ của nó dọc theo bất kì con đường nào đều như nhau. Nói cánh khác, thời gian ánh sáng truyền từ nguồn $A$ đến ảnh $A'$ không phụ thuộc vào đường truyền ánh sáng.
\begin{enumerate}
  \item Cho một gương cầu lõm có bán kính $R$, tâm hình học $O$, quang tâm $C$ và quang trục $OC$ (Hình 3a). Bằng cách sử dụng nguyên lý đẳng thời, hãy chỉ ra rằng tất cả các tia $A'A$ song song với quang trục sau khi phản xạ trên gương sẽ giao nhau tại một điểm $F$ được gọi là tiêu điểm. Tìm tiêu cự $f=FC$ của gương.
  \item Cho một thấu kính phẳng lồi có bán kính cong của mặt lồi là $R$ và chiết suất của vật liệu làm thấu kính là $n$. $O$ là tâm hình học của mặt lồi và $C$ là quang tâm của thấu kính. $OC$ là quang trục của thấu kính (hình 3b). Bằng cách sử dụng nguyên lý đẳng thời, hãy chỉ ra rằng tất cả các tia $A'A$ song song với quang trục sau khi phản xạ trên gương sẽ giao nhau tại một điểm $F$ được gọi là tiêu điểm. Tìm tiêu cự $f=FC$ của thấu kính.
  \item Trên quang trục của một gương cầu lõm bán kính $R$ có đặt một nguồn sáng điểm $A$ ở khoảng cách $a=AC$ tính từ quang tâm của gương. Ảnh của điểm sáng này nằm ở khoảng cách $b=BC$ tính từ quang tâm của gương (hình 3c).
        \begin{enumerate}
          \item[a.] Sử dụng nguyên lý đẳng thời, chứng minh ảnh của điểm sáng này là ảnh thật.
          \item[b.] Thiết lập biểu thức biểu diễn mối quan hệ giữa $a,b$ và tiêu cự $f$ của gương. Biểu thức mà bạn tìm được gọi là "công thức gương cầu lõm"
        \end{enumerate}
  \item  Như bạn đã biết, các ảnh có thể là ảo (hình 3d).
        \begin{itemize}
          \item[a.] Cải tiến nguyên lý đẳng thời sao cho ta có thể dùng nó để xác định vị trí của các ảnh ảo.
          \item[b.] Chứng minh "công thức gương cầu lồi" có thể được viết dưới dạng tương tự "công thức gương cầu lõm" nếu bạn định nghĩa lại các đại lượng trong công thức đó.
          \item[c.] Giải thích nguyên lý đẳng thời bằng các lập luận trong phạm vi không quá 100 từ.
        \end{itemize}
\end{enumerate}


\begin{figure*}
  \centering
  \begin{subfigure}[b]{0.475\textwidth}
    \centering
    \includegraphics[width=0.7\textwidth]{Figures/P3/Fig 3.4.png}
    \caption{}
  \end{subfigure}
  \hfill
  \begin{subfigure}[b]{0.475\textwidth}
    \centering
    \includegraphics[width=0.85\textwidth]{Figures/P3/Fig 3.5.png}
    \caption{}
  \end{subfigure}
  \vskip\baselineskip
  \begin{subfigure}[b]{0.475\textwidth}
    \centering
    \includegraphics[width=0.8\textwidth]{Figures/P3/Fig 3.6.png}
    \caption{}
  \end{subfigure}
  \hfill
  \begin{subfigure}[b]{0.475\textwidth}
    \centering
    \includegraphics[width=0.8\textwidth]{Figures/P3/Fig 3.7.png}
    \caption{}
  \end{subfigure}
  \begin{center}
    \figurename{ 3}
  \end{center}
\end{figure*}

\subsection*{Phần 3: Nguyên lý Fermat}
\begin{enumerate}
  \item Điểm $A$ nằm trong môi trường có chiết suất $n_{1}$, điểm $B$ nằm trong môi trường có chiết suất $n_{2}$. Một tia sáng xuất phát từ $A$, sau khi khúc xạ qua mặt phân cách, sẽ đi đến $B$. Sử dụng nguyên lý Fermat, thiết lập công thức cho định luật khúc xạ ánh sáng.
  \item Cho một ống trụ rỗng có bán kính $R$, mặt trong của lớp trụ được mạ bạc nhờ đó ánh sáng có thể phản xạ trên nó. Khảo sát sự phản xạ của một tia sáng bên trong ống trụ trong mặt phẳng vuông góc với trục đối xứng của ống, chỉ ra tất cả các quỹ đạo khả dĩ của tia sáng $ACB$. Các điểm $A, B, C$ đều nằm trên mặt trong của ống trụ. Tâm của mặt cắt là $O$ và vị trí của điểm $C$ được xác định bởi góc $\phi$.
        \begin{enumerate}
          \item[a.] Xác định biểu thức của chiều dài quỹ đạo $ACB$ theo góc $\phi$ - $L(\phi)$. Phác hoạ đồ thị biểu diễn sự phụ thuộc này cho tất cả các giá trị khả thi của $\phi$.
          \item[b.] Xác định các giá trị của góc $\phi$ ứng với quỹ đạo thực của tia sáng. Chỉ ra các giá trị này trên đồ thị $L-\phi$.
        \end{enumerate}
\end{enumerate}

\begin{figure}[h]
  \centering
  \begin{subfigure}[b]{0.49\textwidth}
    \centering
    \includegraphics[width=0.7\textwidth]{Figures/P3/Fig 3.8.png}
  \end{subfigure}
  \hfill
  \begin{subfigure}[b]{0.49\textwidth}
    \centering
    \includegraphics[width=0.7\textwidth]{Figures/P3/Fig 3.9.png}
  \end{subfigure}
\end{figure}

\begin{enumerate}
  \setcounter{enumi}{2}
  \item Các kết luận:
        \begin{enumerate}
          \item[a.] Cải tiến công thức của nguyên lý Fermat để có thể mô tả tất cả các trường hợp đã khảo sát trong bài này.
          \item[b.] Lí giải bằng lời nguyên lý Fermat trong phạm vi không quá 100 từ.
        \end{enumerate}
\end{enumerate}

\noindent\textbf{Câu IV:}\\
\noindent Một khối băng hình bán cầu có bán kính $R$ và chiết suất $n$ nằm trên một mặt bàn ấm và được đun nóng từ từ. Nhiệt lượng do bàn truyền cho khối băng tỉ lệ với diện tích tiếp xúc giữa chúng. Biết rằng khối băng sẽ nóng chảy hoàn toàn sau thời gian $T_0$. Trong suốt quá trình, một chùm laser được chiếu vào khối băng. Chùm tia được chiếu theo phương vuông góc với bàn và cách trục đối xứng của băng một đoạn $\dfrac{R}{2}$ như hình 4.1.\\
\begin{figure}[h]
  \centering
  \includegraphics[width=1\textwidth]{Figures/Problems/Fig 4.1.png}
  \begin{center}
    \figurename{ 4.1}
  \end{center}
\end{figure}

\indent Giả sử nhiệt độ của khối băng và không khí bao quanh nó là $0^\circ$ trong suốt quá trình đun nóng. Chùm laser không truyền năng lượng cho khối băng. Toàn bộ lượng nước hình thành do sự nóng chảy đều chảy xuống bàn và khối băng không di chuyển trong suốt quá trình.
\begin{enumerate}
  \item Xác định vị trí $x_0$ mà tia laser chạm bàn tại thời điểm $t=0$ theo $n$ và $R$.
  \item Xác định độ cao của khối băng $z(t)$ tại thời điểm $t$ theo $R$ và $T_0$.
  \item Xác định vị trí $x(t)$ mà tia laser chạm bàn tại thời điểm $t\geqslant0$ theo $n, R, T_0$ và $t$.
\end{enumerate}

\noindent\textbf{Câu V:}\\
\begin{wrapfigure}{r}{6cm}
  \centering
  \vspace{-30px}
  \includegraphics[width=0.25\textwidth]{images/Hinh 5.PNG}
  \begin{center}
    \figurename{ 5}
  \end{center}
\end{wrapfigure}

\vspace{-30px}
\noindent Một hạt có khối lượng $m$, điện tích $e$ chuyển động dưới tác dụng của một lưỡng cực điện được đặt cố định tại gốc toạ độ (Hình 5). Momen lưỡng cực $\vec{p}$ hướng dọc theo chiều dương của trục $Oz$. Gọi $\vec{r}(t)$ là vector vị trí của hạt tại thời điểm $t$, $\vec{r}(t)$ có độ lớn $r(t)$ và hợp với momen lưỡng cực $\vec{p}$ một góc $\theta(t)$. Biết $r(t=0)=r_{0}>0$, $\theta(t=0)=\theta_{0}$. Hằng số điện môi trong chân không là $\varepsilon_{0}$ và biểu thức của điện trường và điện thế do lưỡng cực tạo ra có dạng:
\begin{equation*}
  \vec{E}=\frac{3(\vec{p}\cdot\hat{r})\hat{r}-\vec{p}}{4\pi\varepsilon_{0}r^{3}},\quad\varphi=\frac{\vec{p}\cdot\vec{r}}{4\pi\varepsilon_{0}r^{3}}
\end{equation*}
trong đó $\hat{r}$ là vector đơn vị theo hướng $\vec{r}$.
\begin{enumerate}
  \item Giả sử hạt chuyển động trong mặt phẳng vuông góc với trục $Oz$ trên một quỹ đạo tròn quanh trục $Oz$, hãy xác định góc $\theta_{0}$ khi đó và độ lớn vận tốc $v_{0}$ của hạt.
  \item Giả sử tại $t=0$, hạt đứng yên:
        \begin{enumerate}
          \item[a.] Xác định mối liên hệ giữa độ lớn momen động lượng của hạt so với gốc toạ độ $O$ và góc $\theta$ trong quá trình chuyển động sau đó.
          \item[b.] Xác định mối liên hệ giữa độ lớn vận tốc của hạt trên phương hướng tâm $v_{r}$ và $r$ sau đó.
          \item[c.] Tìm điều kiện của $\theta_{0}$ để phạm vi chuyển động của hạt bị giới hạn, không tính đến trường hợp hạt va chạm với lưỡng cực. Trong trường hợp giá trị của $\theta_{0}$ bằng với giá trị tới hạn vừa tìm được, hãy xác định quỹ đạo chuyển động của hạt trong mặt phẳng thẳng đứng.
          \item[d.] Trong trường hợp $\theta_{0}$ không thoả mãn điều kiện tìm được ở ý trên, hãy xác định biểu thức của $r(t)$.
        \end{enumerate}
\end{enumerate}

\noindent\textbf{Câu VI:}\\
\begin{wrapfigure}{r}{8cm}
  \centering
  \vspace{-30px}
  \includegraphics[width=0.45\textwidth]{images/Hinh 6.PNG}
  \vspace{-20px}
  \begin{center}
    \figurename{ 6}
  \end{center}
\end{wrapfigure}

\vspace{-30px}
\noindent Khi một vật dẫn điện được đặt trong từ trường biến thiên, dòng điện Foucault (dòng điện xoáy) sẽ xuất hiện bên trong nó. Dòng Foucault có thể tạo ra tác dụng nhiệt được sử dụng để nung nóng và rèn kim loại; bên cạnh đó, tác dụng cơ của dòng điện này còn được ứng dụng để hãm chuyển động, truyền động, treo các vật thể dẫn điện,\dots. Để nghiên cứu hai tác dụng nêu trên của dòng điện xoáy, ta sẽ sử dụng một thiết bị như hình 6: một cuộn dây bán kính $b$ được đặt cố định trên mặt bàn nằm ngang, mang dòng điện xoay chiều có cường độ $I(t)=I_{0}\cos\omega t$. Một quả cầu dẫn điện có trọng lượng $G$, bán kính $b$, độ dẫn điện $\sigma$ được đặt sao cho tâm quả cầu nằm trên trục đối xứng của vòng dây, cách tâm vòng dây một khoảng $h\ll a$. Giả sử $a\ll b$ và độ sâu bề mặt của quả cầu là $\sigma=\sqrt{\dfrac{2}{\omega\mu_{9}\sigma}}\ll a$ với $\mu_{0}$ là độ từ thẩm trong chân không. Bỏ qua hiện tượng bức xạ điện từ.
\begin{enumerate}
  \item Đặt một quả cầu dẫn lý tưởng $(\sigma)\rightarrow\infty$ bán kính $a$ trong một từ trường đều và ổn định $\vec{B}_{e}=B_{e}\hat{z}$, dòng điện từ hoá bên trong quả cầu sẽ tạo ra một từ trường $\vec{B}'$ bên ngoài quả cầu, từ trường này tương tự với từ trường do một lưỡng cực từ lý tưởng đặt tại tâm quả cầu tạo ra:
        \begin{equation*}
          \vec{B}'=\frac{\mu_{0}}{4\pi}\frac{3(\vec{m}\cdot\hat{r})\hat{r}-\vec{m}}{r^{3}}
        \end{equation*}
        trong đó $\vec{m}$ là momen lưỡng cực từ của lưỡng cực, cũng là momen lưỡng cực của quả cầu dẫn lý tưởng này. Xem $\vec{B}_{e}$ như một đại lượng đã biết, xác định $\vec{m}$ và phân bố dòng điện trên bề mặt quả cầu dẫn.
  \item Xác định cảm ứng từ $\vec{B}$ do dòng điện trong dây dẫn tạo ra tại tâm quả cầu vào thời điểm $t$.
  \item Xác định giá trị của $I_{0}$ để quả cầu có thể cân bằng tại vị trí như hình 6. Giả sử $I_{0}$ đủ lớn để biên độ dao động của quả cầu dẫn quanh vị trí cân bằng có thể bỏ quả. Cho biết: Nếu momen từ $\vec{m}$ của lưỡng cực từ và từ trường ngoài $\vec{B}_{e}$ tại vị trí của lưỡng cực có phương song song với $Oz$, các thành phần tương ứng của chúng là $m$ và $B_{e}=B_{e}(z)$ thì lực tác dụng lên lưỡng cực có dạng:
        \begin{equation*}
          \vec{F}=m\frac{dB_{e}}{dz}\hat{z}
        \end{equation*}
  \item Khi khảo sát tác dụng nhiệt của dòng điện xoáy, ta phải xét đến hiệu ứng bề mặt của quả cầu dẫn. Giả sử dòng điện xoáy được phân bố đều trên phương bán kính trong độ sâu bề mặt và bằng không ở mọi nơi khác bên trong quả cầu. Xác định biểu thức gần đúng của nhiệt năng trung bình do quả cầu toả ra trong một chu kỳ.
\end{enumerate}

\noindent\textbf{Câu VII:}\\
\noindent Tầng thấp nhất của khí quyển là tầng đối lưu, có độ dày khoảng \SI{10}{\kilo\metre}. Trong tầng đối lưu, khi càng lên cao thì nhiệt độ không khí càng giảm. Bên trên tầng đối lưu là tầng bình lưu, trong khoảng \SI{10}{\kilo\metre}, nhiệt độ gần như không thay đổi theo độ cao. Phía trên tầng bình lưu, trong một khoảng độ cao nhất định, nhiệt độ khí quyển lại tăng dần theo độ cao, tầng này được gọi là tầng nghịch (Hình 7).
\begin{enumerate}
  \item Do ánh nắng Mặt Trời, nhiệt độ không khí gần mặt đất cao hơn. Sự thay đổi nhiệt độ khí quyển trong tầng đối lưu có thể xem là kết quả của quá trình biến đổi đoạn nhiệt của không khí. Xác định mối liên hệ giữa nhiệt độ không khí và độ cao tính từ mặt đất. Giả sử nhiệt độ không khí ngay sát mặt đất là $T_{0}$, không khí là khí lý tưởng có khối lượng mol $\mu$, hệ số đoạn nhiệt $\gamma$, và hằng số khí lý tưởng $R$. Gia tốc trọng trường trong tần đối lưu có thể xem là hằng số và bằng $g$.
  \item Vận tốc truyền âm trong không khí dược tính bởi $v_{s}=\sqrt{\left(\dfrac{dp}{d\rho}\right)_{S}}$, trong đó $p$ là áp suất và $\rho$ là khối lượng riêng của không khí. Bỏ qua ảnh hưởng của gió, xác định mối liên hệ giữa vận tốc âm thanh và nhiệt độ $T$ của không khí. Nhiệt độ này thay đổi như thế nào theo độ cao?
  \item Khi sóng âm truyền trong các môi trường có vận tốc âm thanh khác nhau, nó cũng xảy ra hiện tượng phản xạ và khúc xạ, các hiện tượng này xảy ra tương tự với hiện tượng phản xạ và khúc xạ ánh sáng. Để đơn giản hoá, ta có thể coi bề mặt phân cách giữa các lớp khí quyển là phẳng, giả sử nhiệt độ trong tầng đối lưu, tầng nghịch là \SI{-10}{\celsius} và trong tầng bình lưu là \SI{-55}{\celsius}. Trong mô hình này, xét một nguồn âm trong tầng bình lưu, các sóng âm phát ra sẽ đi tới mặt phân cách giữa tầng bình lưu và tầng đối lưu (hay tầng nghịch). Hãy phân tích hiện tượng phản xạ và khúc xạ của sóng âm ứng với các góc tới khác nhau: trong trường hợp nào thì xảy ra khúc xạ?, trong trường hợp nào thì xảy ra phản xạ toàn phần? Đồng thời, vẽ sơ đồ truyền sóng âm cho từng trường hợp. Biết \SI{0}{\celsius}=\SI{273,15}{\kelvin}.
  \item Tiếp tục đơn giản hoá, giả sử nhiệt độ khí quyển không thay đổi theo độ cao, các mặt phân cách là các mặt cầu đồng tâm, nhiệt độ của tầng đối lưu và tầng nghịch giống nhau và đều cao hơn nhiệt độ của tầng bình lưu. Xét một nguồn âm và một khí cầu có gắn một đầu thu âm thanh, cả hai đều có thể ở trong tầng đối lưu hoặc tầng bình lưu. Bỏ quả sự phản xạ sóng âm trên mặt đất cũng như sự hấp thụ năng lượng của sóng âm trong khí quyển. Với 4 trường hợp khác nhau khi nguồn âm và đầu thu nằm lần lượt trong tầng đối lưu và tầng bình lưu, hãy xác định trong trường hợp nào khí cầu có thể phát hiện âm thanh từ khoảng cách xa (vài nghìn \SI{}{\kilo\metre}), và trong trường hợp nào khí cầu chỉ có thể phát hiện âm thanh từ khoảng cách ngắn (dưới \SI{1000}{\kilo\metre}). Giải thích thông qua các tính toán định lượng. Cho biết bán kính Trái Đất và khoảng $R_{e}=$\SI{6371}{\kilo\metre}.
\end{enumerate}

\noindent\textbf{Câu VIII:}\\
\noindent Bằng cách sử dụng quang phổ tia X và phổ quang điện tử, ta có thể thu được thông tin về cấu trúc của vật chất. Sơ đồ bố trí thiết bị thí nghiệm được chỉ ra trong hình 8. Ống tia X bao gồm một cathode của súng điện tử và một anode kim loại. Các electron phát ra từ cathode sẽ được tăng tốc bởi điện áp, sau đó va chạm với anode kim loại để tạo ra tia X (không tính đến bức xạ trước khi electron va chạm với anode). Phổ của nó bao gồm phổ liên tục của bức xạ hãm và phổ đặc trưng rời rạc. Trong thí nghiệm, người ta sử dụng tia X có bước sóng nhất định để tương tác với bia vật liệu, sau đó sử dụng đầu thu quang phổ tia X và đầu thu phổ quang điện tử để phát hiện tia X và quang điện tử phát ra.
\newpage

\begin{figure}[h]
  \centering
  \includegraphics[width=0.65\textwidth]{images/Hinh 8a.png}
  \begin{center}
    \figurename{ 8a: Sơ đồ bố trí thí nghiệm.}
  \end{center}
\end{figure}

\begin{figure}[h]
  \centering
  \includegraphics[width=0.58\textwidth]{images/Hinh 8b.png}
  \begin{center}
    \figurename{ 8b: Quang phổ tia X.}
  \end{center}
\end{figure}

\noindent Biết hằng số Rydberg $R_{\infty}=$\SI{10973731}{\metre^{-1}}, $hc=$\SI{1240}{\nano\metre\electronvolt} trong đó $h$ và $c$ lần lượt là hằng số Plank và vận tốc ánh sáng trong chân không.
\begin{enumerate}
  \item Lý thuyết Bohr có thể lý giải các mức năng lượng và phổ của nguyên tử Hydrogen hoặc hệ ion hoá giống Hydrogen. Năng lượng và trạng thái của electron trong hệ phụ thuộc vào số lượng tử $n$. Đối với hệ nguyên tử gồm nhiều electron, electron trong nguyên tử có thể được chia thành các lớp vỏ $K(n=1), L(n=2)$ và $M(n=3)$ tuỳ thuộc vào giá trị của $n$. BIết rằng năng lượng của electron trong lớp $K$ của nguyên tử kim loại ở anode là \SI{-20,1}{\kilo\electronvolt}.
        \begin{enumerate}
          \item[a.] Xác định điện áp tối thiểu cần sử dụng để ion hoá một electron trong lớp vỏ $K$ của nguyên tử kim loại, từ đó gây ra sự chuyển dịch mức năng lượng của electron từ lớp $L$ về lớp $K$ và phát ra tia X. Kết quả làm tròn đến ba chữ số thập phân.
          \item[b.] Với điện áp tính được ở ý a, bước sóng ngắn nhất của tia X phán ra từ ống tia X là bao nhiêu? Kết quả làm tròn đến ba chữ số thập phân.
          \item[c.] Nếu năng lượng của photon tia X đặc trưng $K_{\alpha}$ phát ra từ nguyên tử kim loại là \SI{17,44}{\kilo\electronvolt}, hãy xác định số điện tích hạt nhân của nguyên tử này.
        \end{enumerate}
  \item Sử dụng tia X đặc trưng $K_{\alpha}$ có bước sóng $\lambda$ tác động vào một bia vật liệu, giả sử electron trong bia vật liệu ở trạng thái tĩnh và có khối lượng nghỉ là $m$. Biết góc tán xạ (góc giữa tia X tán xạ và tia X tới) là $\theta$. Hãy tìm bước sóng của tia X tán xạ và động năng của electron quang điện tử.
  \item Trong thí nghiệm, phổ tán xạ đo được tại góc $\theta=135^{\circ}$. Hình 8b cho thấy có hai đỉnh phổ ứng với các bước sóng $\lambda_{1}$ và $\lambda_{2}$. Giải thích nguồn gốc hai đỉnh phổ này và nguyên nhân khiến cho độ rộng của đỉnh phổ $\lambda_{1}$ lớn hơn độ rộng của đỉnh phổ $\lambda_{2}$.
\end{enumerate}

\newpage

%%%%%%%%%% TITLE#2 %%%%%%%%%%
\begin{center}
  \noindent\Large\textbf{LỜI GIẢI THAM KHẢO}
\end{center}
\vspace{5mm}

%%%%%%%%%% SOLUTIONS %%%%%%%%%%
\setcounter{equation}{0}
\noindent\textbf{Câu I:}\\
\noindent\textbf{1a.}
\begin{equation*}
  M=\frac{\pi r^{2}L}{2}(1+c)
\end{equation*}
\begin{equation*}
  d=\frac{\dfrac{\pi r^{2}L}{2}\left(\dfrac{4r}{3\pi}-\dfrac{4r}{3\pi}c\right)}{\dfrac{\pi r^{2}L}{2}(1+c)}=\frac{4r}{3\pi}\left(\frac{1-c}{1+c}\right)
\end{equation*}

\noindent\textbf{1b.} Momen quán tính của một khối trụ đặc có khối lượng $M$ và bán kính $r$ là $\dfrac{1}{2}Mr^2$ do đó
\begin{equation*}
  I=\frac{1}{2}\left(\frac{\pi r^{2}L}{2}\right)(1+c)r^{2}=\frac{\pi r^{4}L}{4}(1+c)
\end{equation*}

\noindent\textbf{2.} Định luật II Newton cho
\begin{equation*}
  \tau=I\ddot{\theta}
\end{equation*}
momen lực đối với trục đối xứng là
\begin{equation*}
  \begin{gathered}
    \tau=-Mgd\sin\theta \\
    I\ddot{\theta}=-Mgd\sin\theta \\
    \ddot{\theta}=-\frac{Mgd}{I}\sin\theta \\
    \Rightarrow T=\frac{2\pi}{\omega}=2\pi\sqrt{\frac{I}{Mgd}}
  \end{gathered}
\end{equation*}

\noindent\textbf{3a.} \\
\noindent\underline{\textbf{Cách 1}}: Phương trình chuyển động của khối trụ đối với trục quay đi qua điểm tiếp xúc
\begin{equation*}
  \tau=I_{\text{con}}\ddot{\theta}
\end{equation*}
momen lực tác dụng lên khối trụ tương tự như ý trên
\begin{equation*}
  \tau = -Mgd\sin\theta\approx - M g d \theta
\end{equation*}
momen quán tính đối với trục quay qua điểm tiếp xúc được cho bởi
\begin{equation*}
  I_{\text{con}}=I_{cm}+M(r-d)^{2}
\end{equation*}
\begin{equation*}
  I=I_{cm}+Md^{2}
\end{equation*}
\begin{equation*}
  \Rightarrow I_{\text{con}}=I+M((r-d)^{2}-d^{2})=I+M(r^{2}-2rd)
\end{equation*}
\begin{equation*}
  \Rightarrow\ddot{\theta}=-\frac{Mgd}{I_{con}}\theta
\end{equation*}
chu kì dao động
\begin{equation*}
  T=2\pi\sqrt{\frac{I_{con}}{Mgd}}=2\pi\sqrt{\frac{I+M(r^{2}-2rd)}{Mgd}}
\end{equation*}
có thể thấy, khi $d\rightarrow0$, chu kì $T\rightarrow0$.\\

\noindent\underline{\textbf{Cách 2}}: Hàm Lagrange là
\begin{equation*}
  L=T-U=\frac{1}{2}I_{con}\dot{\theta}^2-\frac{1}{2}Mgd\theta^2
\end{equation*}
phương trình chuyển động được cho bởi
\begin{equation*}
  \begin{gathered}
    \frac{d}{dt}\frac{\partial L}{\partial\theta}-\frac{\partial L}{\partial\theta}=0\Rightarrow I_{con}\ddot{\theta}+Mgd\theta=0 \\
    \Rightarrow\ddot{\theta}=-\frac{Mgd}{I_{con}}\theta                                                                           \\
    \Rightarrow T=2\pi\sqrt{\frac{I_{con}}{Mgd}}=2\pi\sqrt{\frac{I+M(r^{2}-2rd)}{Mgd}}
  \end{gathered}
\end{equation*}

\noindent\textbf{3b.} Vì khối trụ chỉ có thể lăn không trượt, năng lượng của nó là bảo toàn. Để thoát khỏi sự dao động, khối trụ phải có đủ động năng để khối tâm có thể lên đến vị trí thế năng cực đại
\begin{equation*}
  \frac{1}{2}Mv_{cm}^{2}+\frac{1}{2}I_{cm}\omega_{0}^{2}+Mg(r-d)=Mg(r+d)
\end{equation*}
\begin{equation*}
  \begin{gathered}
    \Rightarrow\frac{1}{2}M\omega_{0}^{2}(r-d)^{2}+\frac{1}{2}(I-Md^{2})\omega_{0}^{2}=2Mgd \\
    \Rightarrow\omega_{0}=\sqrt{\frac{4Mgd}{M(r-d)^{2}+(I-Md^{2})}}
  \end{gathered}
\end{equation*}
có thể thấy, khi $d\rightarrow0$, $\omega_0\rightarrow0$, điều này có nghĩa không có bất kì cân bằng bền nào trong giới hạn này và khối trụ sẽ tiếp tục di chuyển về một phía.\\

\newpage
\setcounter{equation}{0}
\noindent\textbf{Câu II:}\\
\vspace{-1cm}
\begin{wrapfigure}{r}{7cm}
  \centering
  \includegraphics[width=0.4\textwidth]{Figures/Fig 2S1.jpg}
\end{wrapfigure}

\noindent\textbf{1.} Vì $\sqrt{\sigma/(\rho g)}\ll S$, bề mặt của lớp thuỷ ngân dừng như là phẳng và bán kính của lớp thuỷ ngân lớn hơn rất nhiều so với độ dày của nó. Giả sử thuỷ ngân đã che phủ hoàn toàn đáy trên của hình trụ nhưng không tràn xuống mặt bên của nó. Xét một lớp chất lỏng có độ rồng $L$, cân bằng lực theo phương ngang cho:
\begin{equation*}
  F\cos\theta-F+p_{cp}Lh=\sigma L\cos\theta-\sigma L+p_{cp}Lh=0
\end{equation*}
trong đó $p_{cp}$ là áp suất thuỷ tĩnh trung bình theo độ dày của lớp thuỷ ngân:
\begin{equation*}
  p_{cp}=\frac{\rho gh}{2}
\end{equation*}
do đó:
\begin{equation*}
  \frac{\rho gh^{2}}{2}=\sigma(1-\cos\theta)\implies h=\sqrt{\frac{2\sigma(1-\cos\theta)}{\rho g}}
\end{equation*}
khi thuỷ ngân che phủ toàn bộ đáy trên của hình trụ, thể tích của nó bằng:
\begin{equation*}
  V_{0}=Sh=S\sqrt{\frac{2\sigma(1-\cos\theta)}{\rho g}}
\end{equation*}



\noindent\textbf{2.} Nếu đặt lên lớp thuỷ ngân một hình trụ có khối lượng $m$, áp suất thuỷ tĩnh tại mỗi điểm bên trong lớp thuỷ ngân sẽ tăng lên một lượng $mg/S_{k}$ với $S_{k}$ là diện tích tiếp xúc giữa thuỷ ngân và đáy dưới của hình trụ mà ta đặt lên. Áp suất thuỷ tĩnh trung bình trong lớp thuỷ ngân khi này:
\begin{equation*}
  p_{cp}=\frac{\rho gh}{2}+\frac{mg}{S_{k}}
\end{equation*}
điều kiện cân bằng:
\begin{equation*}
  \sigma(1-\cos\theta)=\frac{\rho gh^{2}}{2}+\frac{mgh}{S_{k}}
\end{equation*}
trong đó $h=V/S_{k}$ vì lớp thuỷ cân gần như là phẳng. Ta có:
\begin{equation*}
  \sigma(1-\cos\theta)=\frac{\rho gV^{2}}{2S_{k}^{2}}+\frac{mgV}{S_{k}^{2}}
\end{equation*}
thuỷ ngân sẽ lấp đầy hoàn toàn khe hở khi $S=S_{k}$, tức:
\begin{equation*}
  m_{1}=\frac{\sigma S^{2}(1-\cos\theta)}{gV}-\rho V^{2}
\end{equation*}

\begin{figure}[h]
  \centering
  \includegraphics[width=0.4\textwidth]{Figures/Fig 2S2.jpg}
\end{figure}

\noindent\textbf{3.} Trong quá trình thuỷ ngân tràn ra bên ngoài, tiếp tuyến của bề mặt thuỷ ngân sẽ quay $90^{\circ}$. Thuỷ ngân sẽ bắt đầu tràn ra khỏi khe hở khi áp suất thuỷ tĩnh vượt quá giá trị cho phép của lực căng bề mặt theo phương ngang. Xét hai trường hợp:
\begin{figure}[h]
  \centering
  \begin{subfigure}[b]{0.49\textwidth}
    \centering
    \includegraphics[width=0.8\textwidth]{Figures/Fig 2S3.jpg}
    \caption{Trường hợp 1}
  \end{subfigure}
  \hfill
  \begin{subfigure}[b]{0.49\textwidth}
    \centering
    \includegraphics[width=0.7\textwidth]{Figures/Fig 2S4.jpg}
    \caption{Trường hợp 2}
  \end{subfigure}
\end{figure}

\begin{enumerate}
  \item Trường hợp 1: $\theta<\dfrac{\pi}{2}$:\\
        Trong trường hợp này, tiếp tuyến của bề mặt thuỷ ngân sẽ không bao giờ nằm ngang, do đó, độ lớn lực căng bề mặt theo phương ngang sẽ đạt giá trị cực đại khi đường tiếp tuyến hợp một góc $\theta$ với mặt bên của hình trụ. Khi đó:
        \begin{equation*}
          F_{max}=\sigma L(1+\sin\theta)
        \end{equation*}
        như đã chỉ ra ở trên, điều kiện cân bằng là:
        \begin{equation*}
          \sigma(1+\sin\theta)=\frac{\rho gV^{2}}{2S^{2}}+\frac{m_{2}gV}{S^{2}}
        \end{equation*}
        suy ra:
        \begin{equation*}
          m_{2}=\frac{\sigma S^{2}(1+\sin\theta)}{gV}-\rho V^{2}
        \end{equation*}
  \item Trường hợp 2: $\theta\geqslant\dfrac{\pi}{2}$:
        Trong trường hợp này, tiếp tuyến của bề mặt thuỷ ngân có thể nằm theo phương ngang. Khi đó, độ lớn lực căng bề mặt theo phương ngang sẽ có giá trị cực đại:
        \begin{equation*}
          F_{max}=2\sigma L
        \end{equation*}
        lập luận tương tự trường hợp 1, ta được:
        \begin{equation*}
          m_{2}=\frac{2\sigma S^{2}}{gV}-\rho V^{2}
        \end{equation*}
\end{enumerate}




\newpage
\setcounter{equation}{0}
\noindent\textbf{Câu III:}\\
\noindent\textbf{1.} Bên trong vật dẫn, điện trường bằng không. Các được sức phải vuông góc với mặt khoét. Do sự tích tụ các điện tích mặt, cường độ điện trường ở phía bên phải điện tích sẽ mạnh hơn phía bên trái do đó mật độ đường sức bên phải sẽ lớn hơn. Cuối cùng, lưu ý rằng điện tích là dương, các đường sức phải hướng ra ngoài điện tích. Với những lập luận trên, ta có thể phác hoạ các đường sức như hình 1.1:
\begin{figure}[h]
  \centering
  \begin{subfigure}[b]{0.49\textwidth}
    \centering
    \includegraphics[width=0.75\textwidth]{Figures/Solutions/Fig 3.1.png}
    \begin{center}
      \figurename{ 3.1}
    \end{center}
  \end{subfigure}
  \hfill
  \begin{subfigure}[b]{0.49\textwidth}
    \centering
    \includegraphics[width=0.85\textwidth]{Figures/Solutions/Fig 3.2.png}
    \begin{center}
      \figurename{ 3.2}
    \end{center}
  \end{subfigure}
\end{figure}

\noindent\textbf{2.} Để giải quyết bài toán này, ta sẽ sử dụng phương pháp ảnh điện với điện tích ảnh $q'<0$ được cách tâm hốc cầu một khoảng $d$ như hình 1.2. Điện thế tại một điểm $P$ nằm trên mặt hốc được cho bởi
\begin{equation*}
  \begin{gathered}
    \frac{1}{4\pi\varepsilon_{0}}\left(\frac{q}{\sqrt{z^{2}+R^{2}-2zR\cos\theta}}+\frac{q^{\prime}}{\sqrt{d^{2}+R^{2}-2dR\cos\theta}}\right)=\phi_{P}(\theta)=0 \\
    \Rightarrow\frac{q^{2}}{z^{2}+R^{2}-2zR\cos\theta}=\frac{{q^{\prime}}^{2}}{d^{2}+R^{2}-2dR\cos\theta} \\
    \Rightarrow q^{2}(d^{2}+R^{2})-{q^{\prime}}^{2}(z^{2}+R^{2})+2R({q^{\prime}}^{2}z-dq^{2})\cos\theta=0
  \end{gathered}
\end{equation*}
điều này phải đúng với mọi $\theta$
\begin{equation*}
  \begin{cases}
    q^{\prime2}z-dq^2=0                     \\
    q^2(d^2+R^2)-{q^{\prime2}}(z^2+R^2)=0 &
  \end{cases}
\end{equation*}
giải ra ta được
\begin{equation*}
  d=\frac{R^2}{z}\quad\text{và}\quad q^{\prime}=-\frac{qR}{z}
\end{equation*}
lực tác dụng lên điện tích điểm có độ lớn
\begin{equation*}
  |F|=\frac{1}{4\pi\varepsilon_{0}}\frac{|qq^{\prime}|}{(d-z)^{2}}=\frac{1}{4\pi\varepsilon_{0}}\frac{q^{2}Rz}{(R^{2}-z^{2})^{2}}
\end{equation*}

\noindent\textbf{3.} Theo định lý công - động năng, vận tốc tại $z=\dfrac{R}{2}$ được cho bởi
\begin{equation*}
  \frac{1}{2}m(v^{2}-0^{2})=\int_{0}^{k/2}F(z)\mathrm{d}z
\end{equation*}
\begin{equation*}
  \begin{gathered}
    v=\sqrt{\frac{2}{m}\int_{0}^{R/2}F(z)\mathrm{d}z}=\sqrt{\frac{q^{2}R}{2\pi\epsilon_{0}m}\int_{0}^{R/2}\frac{z}{(R^{2}-z^{2})^{2}}dz}=\sqrt{\frac{q^{2}}{2\pi\epsilon_{0}mR}\int_{0}^{1/2}\frac{u}{(1-u^{2})^{2}}du}\\
    v=\sqrt{\frac{1}{12\pi\varepsilon_{0}}\frac{q^{2}}{mR}}
  \end{gathered}
\end{equation*}

\newpage
\setcounter{equation}{0}
\noindent\textbf{Câu IV:}\\
\noindent Giả sử các cạnh song song với dây dẫn có chiều dài là $a$, các cạnh còn lại có chiều dài là $b$. Chọn trục $Ox$ như hình vẽ, gọi $x$ là toạ độ của cạnh gần nhất của khung so với dây dẫn, tại thời điểm ban đầu, khoảng cách này bằng $x_{0}$. Giả sử dòng điện trong dây dẫn có cường độ $I$, khi đó, cảm ứng từ do dây dẫn tạo ra ở khoảng cách $r$ là:
\begin{equation*}
  B(I,r)=\frac{\mu_{0}I}{2\pi r}
\end{equation*}
trong đó, $\mu_{0}$ là độ từ thẩm trong chân không. \\

\begin{figure}[h]
  \centering
  \begin{subfigure}[b]{0.49\textwidth}
    \centering
    \includegraphics[width=0.65\textwidth]{Figures/Fig 4S1.jpg}
  \end{subfigure}
  \hfill
  \begin{subfigure}[b]{0.49\textwidth}
    \centering
    \includegraphics[width=0.65\textwidth]{Figures/Fig 4S2.jpg}
  \end{subfigure}
\end{figure}

\noindent Gọi dòng điện cảm ứng trong khung có cường độ $I_{1}$ và có hướng ngược chiều kim đồng hồ. Lực tác dụng lên khung dây là:
\begin{equation*}
  F_{x}=I_{1}aB(I,x+b)-I_{1}aB(I,x)
\end{equation*}
vì $a,b\ll x_{0}$, ta có:
\begin{equation*}
  F_{x}\approx\frac{\mu_{0}I_{1}Ia}{2\pi}\frac{d}{dx}\left(\frac{1}{x}\right)b=-\frac{\mu_{0}I_{1}Iab}{2\pi x^{2}}
\end{equation*}
gọi $R$ là điện trở của khung và $\phi$ là thông lượng từ qua khung, theo định luật Faraday:
\begin{equation*}
  I_{1}=-\frac{\dot{\phi}}{R}
\end{equation*}
vì kích thước của khung dây rất nhỏ so với khoảng cách từ nó đến dây dẫn, ta có thể xem cảm ứng từ là không đổi trên toàn bộ diện tích của khung dây, khi đó từ thông là:
\begin{equation*}
  \phi\approx B(x)ab
\end{equation*}
suy ra:
\begin{equation*}
  I_{1}=-\frac{ab}{R}\frac{dB(x)}{dt}=-\frac{\mu_{0}ab}{2\pi R}\frac{d}{dt}\left(\frac{I}{x}\right)
\end{equation*}
do đó:
\begin{equation*}
  F_{x}=\left(\frac{\mu_{0}ab}{2\pi}\right)^{2}\frac{I}{R}\frac{1}{x^{2}}\frac{d}{dt}\left(\frac{I}{x}\right)
\end{equation*}
vì cường độ dòng điện trong dây dẫn tăng lên rất nhanh, ta có thể bỏ qua chuyển động của khung trong thời gian tăng dòng điện, khi đó:
\begin{equation*}
  F_{x}\approx\left(\frac{\mu_{0}ab}{2\pi}\right)^{2}\frac{I}{Rx^{3}}\frac{dI}{dt}
\end{equation*}
gọi $m$ và $v_{x}$ lần lượt là khối lượng và vận tốc của khung trên phương $x$, ta có:
\begin{equation*}
  m\frac{dv_{x}}{dt}=\left(\frac{\mu_{0}ab}{2\pi}\right)\frac{I}{Rx^{3}}\frac{dI}{dt}
\end{equation*}
suy ra:
\begin{equation*}
  mdv_{x}=\left(\frac{\mu_{0}ab}{2\pi}\right)\frac{IdI}{Rx^{3}}
\end{equation*}
tích phân hai vế ta được:
\begin{equation*}
  mv_{0}=\left(\frac{\mu_{0}abI_{0}}{2\pi}\right)^{2}\frac{1}{2Rx_{0}^{3}}
\end{equation*}
bây giờ, ta sẽ xem xét chuyển động tiếp theo của khung, khi cường độ dòng điện trong dây dẫn đạt giá trị không đổi $I_{0}$, lực tác dụng lên khung là:
\begin{equation*}
  F_{x}=\left(\frac{\mu_{0}abI_{0}}{2\pi}\right)^{2}\frac{1}{Rx^{2}}\frac{d}{dx}\left(\frac{1}{x}\right)=-\left(\frac{\mu_{0}abI_{0}}{2\pi}\right)^{2}\frac{1}{Rx^{4}}\frac{dx}{dt}
\end{equation*}
định luật II Newton:
\begin{equation*}
  m\frac{dv_{x}}{dt}=-\left(\frac{\mu_{0}abI_{0}}{2\pi}\right)^{2}\frac{1}{Rx^{4}}\frac{dx}{dt}
\end{equation*}
suy ra:
\begin{equation*}
  mdv_{x}=-\left(\frac{\mu_{0}abI_{0}}{2\pi}\right)^{2}\frac{dx}{Rx^{4}}
\end{equation*}
tích phân hai vế ta được:
\begin{equation*}
  m(v_{1}-v_{0})=-\frac{2mv_{0}}{3}\implies v_{1}=\frac{v_{0}}{3}
\end{equation*}
vì $v_{1}>0$, khung sẽ chuyển động ra xa dây dẫn, do đó, nó sẽ không bao giờ dừng lại.\\


\newpage
\setcounter{equation}{0}
\noindent\textbf{Câu V:}\\
\noindent\textbf{1.} Khi hạt chuyển động tròn đều trong mặt phẳng vuông góc với trục $Oz$, nó không chịu tác dụng của lực nào trên phương $Oz$, do đó:
\begin{equation}
  \label{eq:51}
  \hat{z}\cdot e\vec{E}=\frac{ep}{4\pi\varepsilon_{0}r_{0}^{3}}(3\cos^{2}\theta_{0}-1)=0
\end{equation}
suy ra:
\begin{equation}
  \label{eq:52}
  \cos\theta_{0}=\pm\frac{\sqrt{3}}{3}
\end{equation}
và vì hạt chuyển động tròn, lực tác dụng lên nó trong mặt phẳng $Oxy$ phải có phương hướng tâm. Phương trình \eqref{eq:52} chỉ có hai nghiệm, đó là:
\begin{equation}
  \label{eq:53}
  F=\frac{-3ep}{4\pi\varepsilon_{0}r_{0}^{3}}\cos\theta_{0}\sin\theta_{0}>0
\end{equation}
trong phương trình \eqref{eq:52} chỉ có một nghiệm thoả mãn yêu cầu $\cos\theta_{0}<0$, tức là ta có:
\begin{equation}
  \label{eq:54}
  \cos\theta_{0}=-\frac{\sqrt{3}}{3} \implies \theta_{0}=\pi-\arccos\frac{\sqrt{3}}{3}
\end{equation}
bán kính quỹ đạo của hạt là $R=r_{0}\sin\theta_{0}$, thay \eqref{eq:53} vào phương trình của định luật II Newton:

\begin{equation}
  \label{eq:55}
  F=m\frac{v_{0}^{2}}{R}
\end{equation}
ta có:
\begin{equation*}
  v_{0}=\sqrt{\frac{FR}{m}}=\sqrt{-\frac{3eq}{4\pi\varepsilon_{0}mr_{0}^{2}}\cos\theta_{0}\sin^{2}\theta_{0}}=\sqrt{-\frac{3eq}{4\pi\varepsilon_{0}r_{0}^{2}}\cos\theta_{0}(1-\cos^{2}\theta_{0})}
\end{equation*}
sử dụng \eqref{eq:54} ta được:
\begin{equation}
  \label{eq:56}
  v_{0}=\sqrt{\frac{\sqrt{3}eq}{6\pi\varepsilon_{0}mr_{0}^{2}}}
\end{equation}

\noindent\text{2a.} Momen động lượng của hạt là:
\begin{equation}
  \label{eq:57}
  \vec{L}=\vec{r}\times m \vec{v}=\vec{r}\times m(\dot{r}\hat{r}+r\dot{\theta}\hat{\theta})=mr^{2}\dot{\theta}\hat{n}
\end{equation}
momen lực tác dụng lên hạt:
\begin{equation}
  \label{eq:58}
  \tau=\vec{r}\times e\vec{E}=-\frac{e}{4\pi\varepsilon_{0}r^{3}}\vec{r}\times\vec{p}=\frac{ep\sin\theta}{4\pi\varepsilon_{0}r^{2}}\hat{n}
\end{equation}
với $\hat{n}=\hat{r}\times\hat{\theta}$. Phương trình chuyển động của hạt:
\begin{equation}
  \label{eq:59}
  \frac{dL}{dt}=\frac{d(mr^{2}\dot{\theta})}{dt}=\frac{ep\sin\theta}{4\pi\varepsilon_{0}r^{2}}
\end{equation}
nhân vế theo vế phương trình \eqref{eq:59} với $L=mr^{2}\dot{\theta}$ ta được:
\begin{equation*}
  L\frac{dL}{dt}=\frac{ep\sin\theta}{4\pi\varepsilon_{0}r^{2}}\cdot mr^{2}\frac{d\theta}{dt}=\frac{mep\sin\theta}{5\pi\varepsilon_{0}}\frac{d\theta}{dt}
\end{equation*}
điều này dẫn tới:
\begin{equation*}
  \frac{dL^{2}}{dt}=-\frac{d}{dt}\left(\frac{mep}{2\pi\varepsilon_{0}}\cos\theta\right)
\end{equation*}
suy ra:
\begin{equation}
  \label{eq:510}
  L^{2}+\frac{mep}{2\pi\varepsilon_{0}}\cos\theta=\text{const}
\end{equation}
hạt đứng yên tại $t=0$, nghĩa là:
\begin{equation}
  \label{eq:511}
  L^{2}=\frac{mep}{2\pi\varepsilon_{0}}(\cos\theta_{0}-\cos\theta)
\end{equation}

\noindent\textbf{2b.} Điện thế do lưỡng cực điện tạo ra:
\begin{equation*}
  \varphi=\frac{\vec{p}\cdot\vec{r}}{4\pi\varepsilon_{0}r^{3}}=\frac{p\cos\theta}{4\pi\varepsilon_{0}r^{2}}
\end{equation*}
bảo toàn năng lượng:
\begin{equation}
  \label{eq:512}
  \frac{1}{2}mv^{2}+e\varphi=\frac{1}{2}m(\dot{r}^{2}+r^{2}\dot{\theta}^{2})+\frac{ep\cos\theta}{4\pi\varepsilon_{0}r^{2}}=\frac{ep\cos\theta_{0}}{4\pi\varepsilon_{0}r_{0}^{2}}
\end{equation}
hay dưới dạng động lượng $p_{r}=mv_{r}$ và momen động lượng $L=mr^{2}\dot{\theta}$:
\begin{equation}
  \label{eq:513}
  \frac{p_{r}^{2}}{2m}+\frac{L^{2}}{2mr^{2}}+\frac{ep\cos\theta}{4\pi\varepsilon_{0}r^{2}}=\frac{ep\cos\theta_{0}}{4\pi\varepsilon_{0}r_{0}^{2}}
\end{equation}
thay \eqref{eq:511} vào \eqref{eq:513} ta được:
\begin{equation*}
  v_{r}^{2}=\frac{ep\cos\theta_{0}}{2\pi m\varepsilon_{0}}\left(\frac{1}{r_{0}^{2}}-\frac{1}{r^{2}}\right)
\end{equation*}
suy ra:
\begin{equation}
  \label{eq:514}
  v_{r}=\sqrt{\frac{ep\cos\theta_{0}}{2\pi m\varepsilon_{0}}\left(\frac{1}{r_{0}^{2}-\frac{1}{r^{2}}}\right)}
\end{equation}

\noindent\textbf{2c.} Vì $v_{r}^{2}\geqslant 0$ nên từ \eqref{eq:514} ta có:
\begin{equation*}
  \text{khi}\theta_{0}<\frac{\pi}{2}, r\geqslant r_{0}
\end{equation*}
lúc này, $v_{r}=\dot{r}=\sqrt{\dfrac{ep\cos\theta_{0}}{2\pi m\varepsilon_{0}}\left(\dfrac{1}{r_{0}^{2}}-\dfrac{1}{r^{2}}\right)}$ tăng khi $r$ tăng. Chuyển động của hạt không bị giới hạn.
\begin{equation*}
  \text{khi}\theta_{0}>\frac{\pi}{2}, r\leqslant r_{0}
\end{equation*}
\begin{equation*}
  \text{khi}\theta_{0}=\frac{\pi}{2}, r=r_{0}
\end{equation*}
trong cả hai trường hợp trên, chuyển động của hạt đều bị giới hạn. Do đó, điều kiện cần tìm là:
\begin{equation}
  \label{eq:515}
  \theta_{0}\geqslant\frac{\pi}{2}
\end{equation}
nếu $\theta_{0}=\dfrac{pi}{2}$, chuyển động của hạt bị giới hạn hoàn toàn, $v_{r}=\dot{r}=0$. Khoảng cách từ hạt đến gốc toạ độ không đổi trong suốt quá trình chuyển động và bằng $r_{0}$; và vì $L\geqslant 0$, phương trình \eqref{eq:511} đòi hỏi:
\begin{equation}
  \label{eq:516}
  L^{2}=-\frac{mep}{2\pi\varepsilon_{0}}\cos\theta\geqslant 0
\end{equation}
như vậy, chuyển động của hạt thoả mãn
\begin{equation}
  \label{eq:517}
  r=r_{0}, \cos\theta\leqslant 0
\end{equation}
có thể thấy, quỹ đạo của nó là một đường bán nguyệt có bán kính $r_{0}$ $\left(\dfrac{\pi}{2}\leqslant\theta\leqslant\dfrac{3\pi}{2}\right)$ trong mặt phẳng thẳng đứng.

\noindent\textbf{2d.} Trong trường hợp chuyển động của hạt không bị giới hạn, phương trình \eqref{eq:514} có thể được viết lại dưới dạng:
\begin{equation}
  \label{eq:518}
  \frac{dr}{dt}=\sqrt{\frac{ep\cos\theta_{0}}{2\pi\varepsilon_{0}m}\left(\frac{1}{r_{0}^{2}}-\frac{1}{r^{2}}\right)}
\end{equation}
hay:
\begin{equation*}
  \sqrt{\frac{ep\cos\theta_{0}}{2\pi\varepsilon_{0}mr_{0}^{2}}dt=\frac{rdr}{\sqrt{r^{2}-r_{0}^{2}}}=d\sqrt{r^{2}-r_{0}^{2}}}
\end{equation*}
qua đó:
\begin{equation}
  \label{eq:519}
  \sqrt{r^{2}-r_{0}^{2}}-\sqrt{\frac{ep\cos\theta_{0}}{2\pi\varepsilon_{0}mr_{0}^{2}}}t=\text{const}
\end{equation}
sử dụng điều kiện bao đầu $r(t=0)=r_{0}>0$, ta có:
\begin{equation*}
  \sqrt{r^{2}-r_{0}^{2}}-\sqrt{\frac{ep\cos\theta_{0}}{2\pi\varepsilon_{0}mr_{0}^{2}}}t=0
\end{equation*}
suy ra:
\begin{equation}
  \label{eq:520}
  r=\sqrt{r_{0}^{2}+\frac{ep\cos\theta_{0}}{2\pi\varepsilon_{0}mr_{0}^{2}}t^{2}}
\end{equation}

\newpage
\setcounter{equation}{0}
\noindent\textbf{Câu VI:}\\
\noindent\textbf{1.} Từ trường bên ngoài quả cầu là:
\begin{equation}
  \label{eq:61}
  \vec{B}=\vec{B}_{e}+\vec{B}'=\vec{B}_{e}+\frac{\mu_{0}}{4\pi r^{3}}\left[3(\vec{m}\cdot\hat{r})\hat{r}-\vec{m}\right]
\end{equation}
đối với một vật dẫn lý tưởng, từ trương bên trong nó bằng không, mặt khác, theo định lý Gauss cho từ trường, thành phân theo phương pháp tuyến của từ trường là liên tục, do đó:
\begin{equation}
  \label{eq:62}
  \hat{r}\cdot\vec{B}\vert_{r=a}=\hat{r}\cdot\vec{B}_{e}+\frac{\mu_{0}}{2\pi a^{3}}(\vec{m}\cdot\hat{r})=0
\end{equation}
do tính đối xứng, $\vec{m}$ phải song song hoặc phản song song với $\vec{B}_{e}$, theo đó, từ \eqref{eq:62} ta có:
\begin{equation}
  \label{eq:63}
  \vec{m}=-\frac{2\pi a^{3}}{\mu_{0}}\vec{B}_{e}
\end{equation}
từ trường  bên ngoài là chồng chập của trường được tạo ra từ dòng điện từ hoá trên quả cầu (tương tự với từ trường tạo ra bởi lưỡng cực từ $\vec{m}$) và từ trường ngoài $\vec{B}_{e}$. Thay \eqref{eq:63} vào \eqref{eq:61} ta được:
\begin{equation}
  \label{eq:64}
  \vec{B}=\vec{B}_{e}-\frac{a^{3}}{2r^{3}}\left[3(\vec{B}_{e}\cdot\hat{r})\hat{r}-\vec{B}_{e}\right]
\end{equation}
theo định lý Ampere, mật độ dòng điện trên bề mặt quả cầu được cho bởi:
\begin{equation}
  \label{eq:65}
  \mu\vec{i}=\hat{r}\times\vec{B}\vert_{r=a}
\end{equation}
thay \eqref{eq:64} vào \eqref{eq:65} ta được:
\begin{equation}
  \label{eq:66}
  \vec{i}=\hat{r}\times\frac{3\vec{B}_{e}}{2\mu_{0}}=\left(-\frac{3B_{e}}{2\mu_{0}\sin\theta}\right)\hat{\phi}
\end{equation}

\noindent\textbf{2.} Do tính đối xứng, cảm ứng từ do dòng điện trong dây dẫn tạo ra tại một điểm nằm trên trục đối xứng của nó có hướng dọc theo trục $Oz$. Sử dụng định luật Biot-Savart-Laplace, ta có:
\begin{equation*}
  \vec{B}=\frac{\mu_{0}I}{4\pi(b^{2}+h^{2})}\frac{b}{\sqrt{b^{2}+h^{2}}}2\pi b\hat{z}
\end{equation*}
hay:
\begin{equation}
  \label{eq:67}
  \vec{B}=\left[\frac{\mu_{0}I_{0}b^{2}}{2(b^{2}+h^{2})^{3/2}}\cos\omega t\right]\hat{z}=B_{0}\cos\omega t\hat{z}
\end{equation}
trong đó:
\begin{equation*}
  B_{0}=\frac{\mu_{0}I_{0}b^{2}}{2(b^{2}+h^{2})^{3/2}}
\end{equation*}

\noindent\textbf{3.} Vì $a\ll b$ nên từ trường do vòng dây tạo ra ở khu vực gần quả cầu dẫn gần như là đều, xấp xỉ từ trường tại tâm quả cầu, được cho bởi phương trình \eqref{eq:67}. Và vì $b\ll a$, quả cầu dẫn có thể xem là lý tưởng và momen lưỡng cực từ của nó có thể tìm được từ phương trình \eqref{eq:63}:
\begin{equation}
  \label{eq:68}
  \vec{m}=-\frac{2\pi a^{3}}{\mu_{0}}\vec{B}
\end{equation}
vì $\vec{m}$ và $\vec{B}$ đều hướng dọc theo trục $Oz$ nên lực tác dụng lên quả cầu là:
\begin{equation*}
  \vec{F}(t)=m\frac{dB}{dz}\hat{z}
\end{equation*}
do đó:
\begin{equation*}
  \vec{F}=-\frac{2\pi a^{3}}{\mu_{0}}B\frac{dB}{dz}\hat{z}=-\frac{\pi a^{3}}{\mu_{0}}\frac{dB^{2}}{dz}\hat{z}
\end{equation*}
thay \eqref{eq:67} vào ta được:
\begin{equation}
  \label{eq:69}
  \vec{F}(t)=-\frac{\pi a^{3}}{\mu_{0}}\frac{d}{dz}\left[\frac{\mu_{0}^{2}I_{0}^{2}b^{4}}{4(b^{2}+z^{2})^{3}}\cos^{2}\omega t\right]=\frac{3\pi a^{3}b^{4}z}{2(b^{2}+z^{2})^{4}}\mu_{0}I_{0}^{2}\cos^{2}\omega t\hat{z}
\end{equation}
mặt khác:
\begin{equation*}
  \langle \cos^{2}\omega t \rangle=\frac{1}{2\pi/\omega}\int_{0}^{2\pi/\omega}\cos^{2}\omega tdt=\frac{1}{2}
\end{equation*}
do đó:
\begin{equation}
  \label{eq:610}
  \langle F \rangle=\frac{1}{2\pi/\omega}\int_{0}^{2\pi/\omega}\vec{F}dt=\frac{3\pi a^{3}b^{4}h}{4(b^{2}+h^{2})^{4}}\mu_{0}I_{0}^{2}\hat{z}
\end{equation}
phương trình cân bằng:
\begin{equation}
  \label{eq:611}
  \langle F \rangle = G
\end{equation}
từ đây ta tìm được:
\begin{equation}
  \label{eq:611}
  I_{0}=\sqrt{\frac{4G(b^{2}+h^{2})^{4}}{3\pi a^{3}b^{4}\mu_{0} h}}
\end{equation}

\noindent\textbf{4.} Thay \eqref{eq:66} vào \eqref{eq:67} ta tìm được  mật đồ dòng điện bề mặt gần đúng trên quả cầu dẫn. Do đó, mật độ dòng điện trong bề dày đặc trưng của quả cầu là:
\begin{equation}
  \label{eq:613}
  \vec{J}=\frac{\vec{i}}{\delta}=\left(-\frac{3B}{2\mu_{0}\delta}\sin\theta\right)\hat{\phi}
\end{equation}
theo định luật Joule, mật độ công suất nhiệt là:
\begin{equation}
  \label{eq:614}
  p=\frac{J^{2}}{\sigma}=\frac{9B^{2}}{4\mu_{0}^{2}\sigma\delta^{2}}\sin^{2}\theta
\end{equation}
do đó, tổng công suất toả nhiệt là:
\begin{equation}
  \label{eq:615}
  P(t)=\int pdV=\int p 2\pi r^{2}\sin\theta drd\theta=\frac{9\pi B^{2}}{2\mu_{0}^{2}\sigma \delta^{2}}\int_{a-\delta}^{a}r^{2}dr\int_{0}^{\pi}\sin^{3}\theta d\theta
\end{equation}
trong đó:
\begin{equation}
  \label{eq:616}
  \int_{a-\delta}^{a}r^{2}dr=\frac{1}{3}[a^{3}-(a-\delta)^{3}]\approx a^{2}\delta
\end{equation}
và:
\begin{equation}
  \label{eq:617}
  \int_{0}^{\pi}\sin^{3}\theta d\theta=\int_{-1}^{1}(1-\cos^{2}\theta)d\cos\theta=\frac{4}{3}
\end{equation}
do đó:
\begin{equation}
  \label{eq:618}
  P(t)=6\pi a^{2}\frac{B^{2}}{\mu_{0}^{2}\sigma\delta}
\end{equation}
thay \eqref{eq:67} và $\delta=\sqrt{\dfrac{2}{\omega\mu_{0}\sigma}}$ vào \eqref{eq:618} ta được:
\begin{equation}
  \label{eq:619}
  P(t)=\frac{3}{4}\frac{\pi a^{2}b^{4}I_{0}^{2}}{(b^{2}+h^{2})^{3}}\sqrt{\frac{2\omega\mu_{0}}{\sigma}}\cos^{2}\omega t
\end{equation}
vì vậy, công suất toả nhiệt trung bình là:
\begin{equation}
  \label{eq:620}
  \langle P \rangle = \frac{3}{8}\frac{\pi a^{2}b^{4}I_{0}^{2}}{(b^{2}+h^{2})^{3}}\sqrt{\frac{2\omega\mu_{0}}{\sigma}}
\end{equation}



\newpage
\setcounter{equation}{0}
\noindent\textbf{Câu VII:}\\
\begin{wrapfigure}[13]{r}{10cm}
  \centering
  \includegraphics[width=0.5\textwidth]{images/Hinh 7a (S).png}
  \begin{center}
    \figurename{ 7a}
  \end{center}
\end{wrapfigure}

\noindent\textbf{1.} Ở độ cao $z$, áp suất khí quyển là $p(z)$, nhiệt độ là $T(z)$, sử dụng phương trình trạng thái khí lý tưởng, ta tìm được khối lượng riêng của không khí là:
\begin{equation}
  \label{eq:71}
  \rho(z)=\frac{n\mu}{V}=\frac{p{z}\mu}{RT(z)}
\end{equation}
trong đó $n$ là số mol, $V$ là thể tích của khí. Như trong hình 7a, trọng lượng của khí nằm trong một hình trụ bán kính đáy $A$, chiều cao $\Delta z$ ở độ cao $z$ là $\rho(z)gA\Delta z$, mặt trên của hình trụ chịu tác dụng của một lực có độ lớn $p(z+\Delta z)A$ trong khi lực tác dụng lên mặt dưới có độ lớn $p(z)A$. Vì khí bay lên từ từ, các lực tác dụng lên nó phải cân bằng với nhau:
\begin{equation}
  \label{eq:72}
  p(z+\Delta z)A+\rho(z)A\Delta zg=p(z)A
\end{equation}
cho $\Delta z\to 0$, phương trình \eqref{eq:72} có thể được viết dưới dạng:
\begin{equation}
  \label{eq:73}
  \frac{dp}{dz}=-\rho g
\end{equation}
sử dụng phương trình \eqref{eq:71}, ta được:
\begin{equation}
  \label{eq:74}
  \frac{dp}{dz}=-\frac{p\mu g}{RT}
\end{equation}
chia cả hai vế cho $p$, ta có:
\begin{equation}
  \label{eq:75}
  \frac{d\ln p}{dz}=-\frac{\mu g}{RT}
\end{equation}
trong quá trình đoạn nhiệt, $pV^{\gamma}$ là hằng số, theo phương trình trạng thái khí lý tưởng, $p^{1-\gamma}T^{\gamma}$ cũng là hằng số, tức:
\begin{equation}
  \label{eq:76}
  C=-(\gamma -1)\ln{p}+\gamma\ln{T}
\end{equation}
lấy đạo hàm hai vế phương trình \eqref{eq:76} theo $z$ ta có:
\begin{equation}
  \label{eq:77}
  0=-(\gamma - 1)\frac{d\ln p}{dz}+\gamma\frac{d\ln T}{dz}
\end{equation}
sử dụng phương trình \eqref{eq:75}:
\begin{equation}
  \label{eq:78}
  \frac{d\ln T}{dz}=\frac{1}{T}\frac{dT}{dz}=\frac{\gamma-1}{\gamma}\frac{d\ln p}{dz}=-\frac{\gamma -1}{\gamma}\frac{\mu g}{RT}
\end{equation}
vì thế:
\begin{equation}
  \label{eq:79}
  \frac{dT}{dz}=-\frac{(\gamma -1)\mu g}{\gamma R}
\end{equation}
sau khi lấy tích phân, ta nhận được:
\begin{equation}
  \label{eq:710}
  T(z)=T_{0}-\frac{(\gamma-1)\mu g}{\gamma R}z
\end{equation}

\noindent\textbf{2.} Đối với quá trình đoạn nhiệt, $pV^{\gamma}$ là hằng số, do đó ta có thể viết $p=C\rho^{\gamma}$, suy ra:
\begin{equation}
  \label{eq:711}
  v_{s}^{2}=\left(\frac{dp}{d\rho}\right)_{S}=\frac{d(C\rho^{\gamma})}{d\rho}=\gamma C\rho^{\gamma-1}=\gamma\frac{C\rho^{\gamma}}{\rho}=\gamma\frac{p}{\rho}
\end{equation}
thay $\rho=\dfrac{n\mu}{V}$ vào phương trình trạng thái khí lý tưởng $pV=nRT$ để có:

\begin{equation}
  \label{eq:712}
  v_{s}^{2}=\gamma\frac{pV}{n\mu}=\gamma\frac{nRT}{n\mu}=\gamma\frac{RT}{\mu}
\end{equation}
Do đó, ở tầng đối lưu, tốc độ truyền âm giảm khi độ cao tăng. Ở tầng bình lưu, nhiệt độ gần như không thay đổi nên tốc độ âm thanh cũng gần như không thay đổi theo độ cao; Ở tầng nghịch, nhiệt độ tăng dần theo độ cao do đó tốc độ âm thanh cũng tăng dần theo độ cao.


\noindent\textbf{3.} Trong mô hình đơn giản được giới thiệu, nhiệt độ khí bên ngoài tầng bình lưu là:
\begin{equation*}
  \SI{-10}{\degreeCelsius}=\SI{263,15}{\kelvin}
\end{equation*}
và nhiệt độ trong tầng bình lưu là:
\begin{equation*}
  \SI{-55}{\degreeCelsius}=\SI{218,15}{\kelvin}
\end{equation*}
do đó, tỉ số giữa vận tốc âm thanh bên ngoài tầng bình lưu và vận tốc âm thanh bên trong tầng bình lưu là:
\begin{equation}
  \label{eq:713}
  \frac{v_{0}}{v_{i}}=\sqrt{\frac{263,15}{218,15}}\approx 1,098
\end{equation}
khi sóng âm trong tầng bình lưu truyền đến mặt phân cách, nếu góc tới bằng $\theta_{i}$ thì theo định luật khúc xạ, góc khúc xạ $\theta_{0}$ phải thoả mãn:
\begin{equation}
  \label{eq:714}
  \sin\theta_{0}=\frac{v_{0}}{v_{i}}\sin\theta_{i}\approx 1,098\sin\theta_{i}
\end{equation}
Từ biểu thức trên, ta nhận thấy có một góc tới tới hạn:
\begin{equation}
  \label{eq:715}
  \theta_{i}^{m}=\arcsin\frac{v_{i}}{v_{0}}\approx 65,61^{\circ}
\end{equation}
khi góc tới nhỏ hơn $\theta_{i}^{m}$ thì định luật khúc xạ được thoả mãn và hiện tượng khúc xạ có thể xảy ra, khi góc tói lớn hơn $\theta_{i}^{m}$ thì định luật khúc xạ không được thoả mãn do đó sẽ xảy ra hiện tượng phản xạ toàn phần. Do đó, biểu hiện của khúc xạ và phản xạ là góc tới nhỏ hơn hay lớn hơn $\theta_{i}^{m}$. Như được biểu diễn bằng đường nét đứt trên hình 7b, khi góc tới nhỏ hơn $\theta_{i}^{m}$, sóng âm có thể phản xạ hoặc khúc xạ cùng lúc, nhờ đó âm thanh có thể được truyền vào tầng nghịch. Như được biểu diễn bằng nét liền trên hình 7b, khi góc tới lớn hơn $\theta_{i}^{m}$, sóng âm sẽ bị phản xạ toàn phần, do đó âm thanh không thể truyền vào tầng ngịch.\\

\begin{figure}[h]
  \centering
  \includegraphics[width=0.53\textwidth]{images/Hinh 7b (S).png}
\end{figure}

\noindent \textbf{4.} Khi khoảng cách là tương đối lớn, ta cần xét đến hình dạng của Trái Đất. Ở đây ta có thể chia vấn đề thành 4 trường hợp, tuỳ thuộc vào vị trí của nguồn âm và đầu thu:
\begin{enumerate}
  \item[A.] Khi nguồn âm và đầu thu đều nằm trong tầng đối lưu, sẽ không có hiện tượng phản xạ toàn phần xảy ra. Vì không có hiện tượng phản xạ, đầu thu chỉ có thể thu được âm thanh nếu không bị Trái Đất cản trở, đường đi của sóng âm được biểu diễn bằng mũi tên $A$ trong hình 7c. Trong trường hợp này, khoảng cách phát hiện tối đa là:
        \begin{equation*}
          l_{1}=\sqrt{(R_{e}+h_{1})^{2}-R_{e}^{2}}\approx2\sqrt{2R_{e}h_{1}}\approx\SI{714}{\kilo\metre}
        \end{equation*}
        Vì vậy, trong trường hợp này, ta không thể phát hiện được nguồn âm ở cách xa hàng nghìn kilomet.
  \item[B.] Nguồn âm ở tầng đối lưu và đầu thu ở tầng bình lưu, do không xảy ra hiện tượng phản xạ, ta chỉ có thể thu được âm khi không bị Trái Đất cản trở, đường đi của sóng âm trong trường hợp này được biểu diễn bằng mũi tên $B$ trong hình 7c. Trong trường hợp này, khoảng cách phát hiện xa nhất sẽ không vượt quá khoảng cách từ một điểm trên đỉnh của tầng dối lưu, dọc theo tiếp tuyến của mặt phân cách đến một điểm trên đỉnh tầng bình lưu, nghĩa là nó sẽ không vượt quá:
        \begin{equation*}
          l_{1}=\sqrt{(R_{e}+h_{2})^{2}-R_{e}^{2}}-\sqrt{(R_{e}+h_{1})^{2}-R_{e}^{2}}\approx2\sqrt{2R_{e}h_{2}} + 2\sqrt{2R_{e}h_{1}}\approx\SI{862}{\kilo\metre}
        \end{equation*}
        Vì vây, trong trường hợp này, ta không thể phát hiện được nguồn âm ở cách xa hàng nghìn kilomet.
  \item[C.] Nguồn âm và đầu thu cùng ở tầng bình lưu. Vì khoảng cách giữa hai bề mặt khí quyển rất nhỏ so với bán kính Trái Đất nên sóng âm bị phản xạ toàn phần trên mặt này cũng sẽ phản xạ toàn phần trên bề mặt kia. Do đó, âm thanh có góc tới nhỏ trong tầng bình lưu có thể bị phản xạ liên tục quả hai mặt phân cách và sẽ bị giới hạn trong tầng bình lưu, do đó nó sẽ tiếp tục là truyền như được biểu diễn trên hình 7d. Vì vậy, trong trường hợp này ta có thể thu được âm thanh ở cách xa vài nghìn kilomet.
  \item[D.] Nguồn âm ở tầng bình lưu và khí cầu ở tầng đối lưu, tương tự như trường hợp $A$ và $B$, ta không thể thu được âm thanh ở cách xa vài nghìn kilomet.
\end{enumerate}

\begin{figure}[h]
  \centering
  \begin{minipage}{6cm}
    \centering
    \includegraphics[width=1.1\textwidth]{images/Hinh 7c (S) .png}
    \begin{center}
      \figurename{ 7c}
    \end{center}
  \end{minipage}
  \hfil
  \begin{minipage}{6cm}
    \centering
    \includegraphics[width=1\textwidth]{images/Hinh 7d (S).png}
    \vspace{1mm}
    \begin{center}
      \figurename{ 7d}
    \end{center}
  \end{minipage}
\end{figure}








\newpage
\setcounter{equation}{0}
\noindent\textbf{Câu VIII:}\\
\noindent\textbf{1a.} Quá trình tạo ra tia X dòng K có thể được mô tả như sau: Electron va chạm với nguyên tử bia ở cực dương làm ion hoá một electron trong lớp K của vỏ nguyên tử bia, dẫn đến một chỗ trống trong lớp K của nguyên tử bia, electron ở lớp L ngay lập tức nhảy sang lớp K và đồng thời phát ra tia X đặc trưng $K_{\alpha}$. Giả sử động năng của các electron sinh ra ở cực âm sau khi được gia tốc bởi điện trường là:
\begin{equation}
  \label{eq:81}
  T_{e}=eU
\end{equation}
năng lượng ion hoá của các electron ở lớp K của nguyên tử bia là:
\begin{equation*}
  E_{k}=\SI{20,1}{\kilo\electronvolt}
\end{equation*}
do đó, điều kiện để một chỗ trống xuất hiện trong lớp K của nguyên tử bia là:
\begin{equation}
  \label{eq:82}
  T_{e}\geqslant E_{k}
\end{equation}
từ \eqref{eq:81} và \eqref{eq:82}, điện áp tối thiểu cần sử dụng bằng:
\begin{equation}
  \label{eq:83}
  U_{min}=\frac{E_{k}}{e}=\SI{20,1}{\kilo\electronvolt}
\end{equation}

\noindent\textbf{1b.} Tia X liên tục phát ra từ ống tia X có nguồn gốc từ bức xạ hãm do các electron va chạm với anode của tấm kim loại tạo ra. Dưới điện áp tối thiểu $U_{min}$, giả sử bước sóng ngắn nhất của tia X phát ra là $\lambda_{min}$, ta có:
\begin{equation}
  \label{eq:84}
  T_{e}h=\nu_{max}=\frac{hc}{\lambda_{min}}
\end{equation}
từ \eqref{eq:81} và \eqref{eq:84}:
\begin{equation}
  \label{eq:85}
  \lambda_{min}=\frac{hc}{eU}\approx\SI{0,0617}{\nano\metre}
\end{equation}

\noindent\textbf{8c.} Năng lượng của tia X đặc trưng $K_{\alpha}$ do nguyên tử kim loại phát ra là $E=\SI{17,44}{\kilo\electronvolt}$, nghĩa là:
\begin{equation}
  \label{eq:86}
  E=\frac{hc}{\lambda_{K\alpha}}=\SI{17,44}{\kilo\electronvolt}
\end{equation}
gọi $\lambda_{K\alpha}$ là bước sóng của tia X đặc trưng $K_{\alpha}$. Đặc điểm cấu trúc quang phổ của tia X dãy K tạo ra do sự va chạm của electron và nguyên tử bia kim loại có liên quan đến tính chất tia của nguyên tử hydrogen. Các quang phổ của hệ thống tương tự như nhau, sự khác biệt nằm ở các điện tích hạt nhân khác nhau mà electron trong nguyển tử cảm nhận được. Vì có một electron sau khi làm xuất hiện một lỗ trống trên lớp vỏ $K$ của nguyên tử bia, nên electron này có tác dụng che chắn hạt nhân. Số điện tích hạt nhân mà electron lớp $L$ cảm nhận được là một giá trị lớn hơn $Z-1$ và nhỏ hơn $Z$. Thông thường, nó được gọi là điện tích hạt nhân hiệu dụng $Z^{*}$. Theo lý thuyết Bohr, tương tự như phổ hệ thống của nguyên tử hydrogen, bước sóng của tia X đặc trưng $K_{\alpha}$ phát ra từ bia kim loại thoả mãn hệ thức:
\begin{equation}
  \label{eq:87}
  \frac{1}{\lambda_{K\alpha}}=R_{\infty}Z^{*2}\left(\frac{1}{1^{2}}-\frac{1}{2^2}\right)
\end{equation}
trong đó $R_{\infty}$ là hằng số Rydgerd và $Z^{*}$ là điện tích hạt nhân hiệu dụng. Từ \eqref{eq:86} và \eqref{eq:87} ta có:
\begin{equation}
  \label{eq:88}
  Z^{*}=\sqrt{\frac{4}{3R_{\infty}\lambda_{K\alpha}}}\sqrt{\frac{4E}{3hcR_{\infty}}}=\approx 41,35
\end{equation}
vì $Z-1<Z^{*}<Z$ nên
\begin{equation*}
  Z^{*}<Z<Z^{*}+1
\end{equation*}
từ \eqref{eq:88} ta có:
\begin{equation}
  \label{eq:89}
  Z=42
\end{equation}

\noindent\textbf{2.} Năng lượng liên kết của các electron trong bia kim loại tương đối nhỏ so với năng lượng của các photon tia X và sự tương tác của tia $X$ và bia kim loại có thể xem như sự tán xạ của tia X và electron tự do. Gọi năng lượng của photon tới là $E_{0}$, động lượng là $\vec{p}_{0}$ và electron bia ban đầu đứng yên. Sau khi photon tia X va chạm với electron bia, năng lượng của electron phát ra là $E_{1}$, động lượng là $\vec{p}_{1}$, góc giữa nó và photon tới là $\theta$; quang điện tử phát ra có năng lượng $E_{2}$, động lượng $\vec{p}_{2}$. Vì năng lượng của photon X rất cao nên động năng của các electron phát ra cũng rất cao, do đó ta phải xét tới hiệu ứng tương đối tính. Năng lượng của quang điện tử phát ra là:
\begin{equation}
  \label{eq:810}
  E_{2}=\frac{mc^{2}}{\sqrt{1-\dfrac{v^{2}}{c^{2}}}}
\end{equation}
động năng:
\begin{equation}
  \label{eq:811}
  T_{2}=E_{2}-mc^{2}
\end{equation}
động lượng là:
\begin{equation}
  \label{eq:812}
  \vec{p}_{2}=\frac{m\vec{v}}{\sqrt{1-\dfrac{v^{2}}{c^{2}}}}
\end{equation}
từ \eqref{eq:811} và \eqref{eq:812} ta có:
\begin{equation}
  \label{eq:813}
  E_{2}^{2}=m^{2}c^{4}+p_{2}^{2}c^{2}
\end{equation}
bảo toàn động lượng:
\begin{equation}
  \label{eq:814}
  \vec{p}_{0}=\vec{p}_{1}+\vec{p}_{2}
\end{equation}
như vậy:
\begin{equation}
  \label{eq:815}
  p_{2}^{2}=p_{0}^{2}+p_{1}^{2}-2p_{0}p_{1}\cos\theta
\end{equation}
bảo toàn năng lượng:
\begin{equation}
  \label{eq:816}
  E_{0}+mc^{2}=E_{1}+\sqrt{m^{2}c^{4}+p_{2}^{2}c^{2}}
\end{equation}
động năng của electron sau va chạm là:
\begin{equation}
  \label{eq:817}
  T_{2}=\sqrt{m^{2}c^{4}+p_{2}^{2}c^{2}}-mc^{2}=E_{0}-E_{1}=c(p_{0}-p_{1})
\end{equation}
suy ra:
\begin{equation}
  \label{eq:818}
  p_{2}^{2}=(p_{0}-p_{1})^{2}+2mc(p_{0}-p_{1})
\end{equation}
từ \eqref{eq:815} và \eqref{eq:816}:
\begin{equation}
  \label{eq:819}
  mc(p_{0}-p_{1})=p_{0}p_{1}(1-\cos\theta)=2p_{0}p_{1}\sin^{2}\frac{\theta}{2}
\end{equation}
thay $\lambda_{0}=\dfrac{h}{p_{0}}$ và $\lambda_{1}=\dfrac{h}{p_{1}}$ vào \eqref{eq:819} ta có:
\begin{equation}
  \label{eq:820}
  \lambda_{1}=\lambda_{0}-2\lambda_{C}\sin^{2}\frac{\theta}{2}
\end{equation}
trong đó $\lambda_{C}$ được gọi là bước sóng Compton của electron:
\begin{equation}
  \label{eq:821}
  \lambda_{C}=\frac{h}{mc}=2,426.10^{-12}\SI{ }{\meter}
\end{equation}
có thể thấy, sự chênh lệch bước sóng là:
\begin{equation*}
  \Delta\lambda=\lambda_{0}-\lambda_{1}=2\lambda_{C}\sin^{2}\frac{\theta}{2}
\end{equation*}
theo định luật bảo toàn năng lượng, động năng của quang điện tử là:
\begin{equation}
  \label{eq:822}
  T_{2}=E_{0}-E_{1}=\frac{hc}{\lambda_{0}}-\frac{hc}{\lambda_{1}}=\frac{hc(1-\cos\theta)}{\lambda_{0}\left(1-\cos\theta+\dfrac{\lambda_{0}}{\lambda_{C}}\right)}
\end{equation}

\noindent\textbf{4.} Đỉnh phổ $\lambda_{2}$ bắt nguồn từ quá trình tán xạ gần như đàn hồi của các photon tia X với các nguyên tử bia. Năng lượng giật lùi của các nguyên tử bia là không đáng kể và bước sóng ánh sáng tán xạ gần như không thay đổi.\\
\indent Đỉnh phổ $\lambda_{1}$ bắt nguồn từ quá trình ion hoá do sự va chạm của photon tia X với các electron bia, do mất mát năng lượng, bước sóng của photon tán xạ trở nên dài hơn.\\
\indent Đỉnh phổ $\lambda_{2}$ rộng hơn vì sự chuyển động của các electron trong bia kim loại. Hình dạng của quang phổ sẽ phản ánh đặc tính phân bố động lượng của các electron bia.\\


\end{document}