\noindent\textbf{1a.} Quá trình tạo ra tia X dòng K có thể được mô tả như sau: Electron va chạm với nguyên tử bia ở cực dương làm ion hoá một electron trong lớp K của vỏ nguyên tử bia, dẫn đến một chỗ trống trong lớp K của nguyên tử bia, electron ở lớp L ngay lập tức nhảy sang lớp K và đồng thời phát ra tia X đặc trưng $K_{\alpha}$. Giả sử động năng của các electron sinh ra ở cực âm sau khi được gia tốc bởi điện trường là:
\begin{equation}
  \label{eq:81}
  T_{e}=eU
\end{equation}
năng lượng ion hoá của các electron ở lớp K của nguyên tử bia là:
\begin{equation*}
  E_{k}=\SI{20,1}{\kilo\electronvolt}
\end{equation*}
do đó, điều kiện để một chỗ trống xuất hiện trong lớp K của nguyên tử bia là:
\begin{equation}
  \label{eq:82}
  T_{e}\geqslant E_{k}
\end{equation}
từ \eqref{eq:81} và \eqref{eq:82}, điện áp tối thiểu cần sử dụng bằng:
\begin{equation}
  \label{eq:83}
  U_{min}=\frac{E_{k}}{e}=\SI{20,1}{\kilo\electronvolt}
\end{equation}

\noindent\textbf{1b.} Tia X liên tục phát ra từ ống tia X có nguồn gốc từ bức xạ hãm do các electron va chạm với anode của tấm kim loại tạo ra. Dưới điện áp tối thiểu $U_{min}$, giả sử bước sóng ngắn nhất của tia X phát ra là $\lambda_{min}$, ta có:
\begin{equation}
  \label{eq:84}
  T_{e}h=\nu_{max}=\frac{hc}{\lambda_{min}}
\end{equation}
từ \eqref{eq:81} và \eqref{eq:84}:
\begin{equation}
  \label{eq:85}
  \lambda_{min}=\frac{hc}{eU}\approx\SI{0,0617}{\nano\metre}
\end{equation}

\noindent\textbf{8c.} Năng lượng của tia X đặc trưng $K_{\alpha}$ do nguyên tử kim loại phát ra là $E=\SI{17,44}{\kilo\electronvolt}$, nghĩa là:
\begin{equation}
  \label{eq:86}
  E=\frac{hc}{\lambda_{K\alpha}}=\SI{17,44}{\kilo\electronvolt}
\end{equation}
gọi $\lambda_{K\alpha}$ là bước sóng của tia X đặc trưng $K_{\alpha}$. Đặc điểm cấu trúc quang phổ của tia X dãy K tạo ra do sự va chạm của electron và nguyên tử bia kim loại có liên quan đến tính chất tia của nguyên tử hydrogen. Các quang phổ của hệ thống tương tự như nhau, sự khác biệt nằm ở các điện tích hạt nhân khác nhau mà electron trong nguyển tử cảm nhận được. Vì có một electron sau khi làm xuất hiện một lỗ trống trên lớp vỏ $K$ của nguyên tử bia, nên electron này có tác dụng che chắn hạt nhân. Số điện tích hạt nhân mà electron lớp $L$ cảm nhận được là một giá trị lớn hơn $Z-1$ và nhỏ hơn $Z$. Thông thường, nó được gọi là điện tích hạt nhân hiệu dụng $Z^{*}$. Theo lý thuyết Bohr, tương tự như phổ hệ thống của nguyên tử hydrogen, bước sóng của tia X đặc trưng $K_{\alpha}$ phát ra từ bia kim loại thoả mãn hệ thức:
\begin{equation}
  \label{eq:87}
  \frac{1}{\lambda_{K\alpha}}=R_{\infty}Z^{*2}\left(\frac{1}{1^{2}}-\frac{1}{2^2}\right)
\end{equation}
trong đó $R_{\infty}$ là hằng số Rydgerd và $Z^{*}$ là điện tích hạt nhân hiệu dụng. Từ \eqref{eq:86} và \eqref{eq:87} ta có:
\begin{equation}
  \label{eq:88}
  Z^{*}=\sqrt{\frac{4}{3R_{\infty}\lambda_{K\alpha}}}\sqrt{\frac{4E}{3hcR_{\infty}}}=\approx 41,35
\end{equation}
vì $Z-1<Z^{*}<Z$ nên
\begin{equation*}
  Z^{*}<Z<Z^{*}+1
\end{equation*}
từ \eqref{eq:88} ta có:
\begin{equation}
  \label{eq:89}
  Z=42
\end{equation}

\noindent\textbf{2.} Năng lượng liên kết của các electron trong bia kim loại tương đối nhỏ so với năng lượng của các photon tia X và sự tương tác của tia $X$ và bia kim loại có thể xem như sự tán xạ của tia X và electron tự do. Gọi năng lượng của photon tới là $E_{0}$, động lượng là $\vec{p}_{0}$ và electron bia ban đầu đứng yên. Sau khi photon tia X va chạm với electron bia, năng lượng của electron phát ra là $E_{1}$, động lượng là $\vec{p}_{1}$, góc giữa nó và photon tới là $\theta$; quang điện tử phát ra có năng lượng $E_{2}$, động lượng $\vec{p}_{2}$. Vì năng lượng của photon X rất cao nên động năng của các electron phát ra cũng rất cao, do đó ta phải xét tới hiệu ứng tương đối tính. Năng lượng của quang điện tử phát ra là:
\begin{equation}
  \label{eq:810}
  E_{2}=\frac{mc^{2}}{\sqrt{1-\dfrac{v^{2}}{c^{2}}}}
\end{equation}
động năng:
\begin{equation}
  \label{eq:811}
  T_{2}=E_{2}-mc^{2}
\end{equation}
động lượng là:
\begin{equation}
  \label{eq:812}
  \vec{p}_{2}=\frac{m\vec{v}}{\sqrt{1-\dfrac{v^{2}}{c^{2}}}}
\end{equation}
từ \eqref{eq:811} và \eqref{eq:812} ta có:
\begin{equation}
  \label{eq:813}
  E_{2}^{2}=m^{2}c^{4}+p_{2}^{2}c^{2}
\end{equation}
bảo toàn động lượng:
\begin{equation}
  \label{eq:814}
  \vec{p}_{0}=\vec{p}_{1}+\vec{p}_{2}
\end{equation}
như vậy:
\begin{equation}
  \label{eq:815}
  p_{2}^{2}=p_{0}^{2}+p_{1}^{2}-2p_{0}p_{1}\cos\theta
\end{equation}
bảo toàn năng lượng:
\begin{equation}
  \label{eq:816}
  E_{0}+mc^{2}=E_{1}+\sqrt{m^{2}c^{4}+p_{2}^{2}c^{2}}
\end{equation}
động năng của electron sau va chạm là:
\begin{equation}
  \label{eq:817}
  T_{2}=\sqrt{m^{2}c^{4}+p_{2}^{2}c^{2}}-mc^{2}=E_{0}-E_{1}=c(p_{0}-p_{1})
\end{equation}
suy ra:
\begin{equation}
  \label{eq:818}
  p_{2}^{2}=(p_{0}-p_{1})^{2}+2mc(p_{0}-p_{1})
\end{equation}
từ \eqref{eq:815} và \eqref{eq:816}:
\begin{equation}
  \label{eq:819}
  mc(p_{0}-p_{1})=p_{0}p_{1}(1-\cos\theta)=2p_{0}p_{1}\sin^{2}\frac{\theta}{2}
\end{equation}
thay $\lambda_{0}=\dfrac{h}{p_{0}}$ và $\lambda_{1}=\dfrac{h}{p_{1}}$ vào \eqref{eq:819} ta có:
\begin{equation}
  \label{eq:820}
  \lambda_{1}=\lambda_{0}-2\lambda_{C}\sin^{2}\frac{\theta}{2}
\end{equation}
trong đó $\lambda_{C}$ được gọi là bước sóng Compton của electron:
\begin{equation}
  \label{eq:821}
  \lambda_{C}=\frac{h}{mc}=2,426.10^{-12}\SI{ }{\meter}
\end{equation}
có thể thấy, sự chênh lệch bước sóng là:
\begin{equation*}
  \Delta\lambda=\lambda_{0}-\lambda_{1}=2\lambda_{C}\sin^{2}\frac{\theta}{2}
\end{equation*}
theo định luật bảo toàn năng lượng, động năng của quang điện tử là:
\begin{equation}
  \label{eq:822}
  T_{2}=E_{0}-E_{1}=\frac{hc}{\lambda_{0}}-\frac{hc}{\lambda_{1}}=\frac{hc(1-\cos\theta)}{\lambda_{0}\left(1-\cos\theta+\dfrac{\lambda_{0}}{\lambda_{C}}\right)}
\end{equation}

\noindent\textbf{4.} Đỉnh phổ $\lambda_{2}$ bắt nguồn từ quá trình tán xạ gần như đàn hồi của các photon tia X với các nguyên tử bia. Năng lượng giật lùi của các nguyên tử bia là không đáng kể và bước sóng ánh sáng tán xạ gần như không thay đổi.\\
\indent Đỉnh phổ $\lambda_{1}$ bắt nguồn từ quá trình ion hoá do sự va chạm của photon tia X với các electron bia, do mất mát năng lượng, bước sóng của photon tán xạ trở nên dài hơn.\\
\indent Đỉnh phổ $\lambda_{2}$ rộng hơn vì sự chuyển động của các electron trong bia kim loại. Hình dạng của quang phổ sẽ phản ánh đặc tính phân bố động lượng của các electron bia.\\
