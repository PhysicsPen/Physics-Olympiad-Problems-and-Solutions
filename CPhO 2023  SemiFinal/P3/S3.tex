\noindent\textbf{1a. } Gọi vận tốc và gia tốc của nêm so với mặt đất là $\vec{v}_{0}$ và $\vec{a}_{0}$, vận tốc và gia tốc của quả cầu so với nêm là $\vec{v}_{C}$ và $\vec{a}_{C}$. Trong hệ quy chiếu gắn với nêm, quả cầu chịu tác dụng của lực quán tính tịnh tiến, trọng lực, phản lực và lực ma sát; phương trình chuyển động của quả cầu là:
\begin{equation}
  \label{eq:31}
  ma_{C}=ma_{0}\cos\theta+mg\sin\theta-f
\end{equation}
đối với chuyển động quay:
\begin{equation}
  \label{eq:32}
  I_{C}\ddot{\varphi}=fr
\end{equation}
gọi $\varphi$ là góc quay của quả cầu so với khối tâm của nó, điều kiện lăn không trượt:
\begin{equation}
  \label{eq:33}
  v_{C}=r\dot{\varphi}
\end{equation}
\begin{equation}
  \label{eq:34}
  a_{C}=r\ddot{\varphi}
\end{equation}
momen quán tính của quả cầu đối với khối tâm của nó là:
\begin{equation}
  \label{eq:35}
  I_{C}=\frac{2}{5}mr^{2}
\end{equation}
thay \eqref{eq:34} và \eqref{eq:35} vào \eqref{eq:32} ta được:
\begin{equation}
  \label{eq:36}
  f=\frac{2}{5}ma_{C}
\end{equation}
bảo toàn động lượng theo phương ngang:
\begin{equation}
  \label{eq:37}
  m(v_{C}\cos\theta-v_{0})=Mv_{0}
\end{equation}
đạo hàm hai vế phương trình \eqref{eq:37} theo thời gian ta được:
\begin{equation}
  \label{eq:38}
  m(a_{C}\cos\theta-a_{0})=Ma_{0}
\end{equation}
từ \eqref{eq:31}, \eqref{eq:36} và \eqref{eq:38} ta tìm được:
\begin{equation}
  \label{eq:39}
  a_{0}=\frac{\dfrac{5}{7}mg\sin\theta\cos\theta}{M+m-\dfrac{5}{7}m\cos^{2}\theta}
\end{equation}

\noindent\textbf{1b.} Thay \eqref{eq:39} vào \eqref{eq:38} ta được:
\begin{equation}
  \label{eq:310}
  a_{C}=\frac{\dfrac{5}{7}(M+m)g\sin\theta}{M+m-\dfrac{5}{7}m\cos^{2}\theta}
\end{equation}

\noindent\textbf{1c.} Xét hệ gồm nêm và quả cầu, thành phần vận tốc khối tâm của hệ theo phương thẳng đứng (chiều dương hướng lên) là:
\begin{equation}
  \label{eq:311}
  v_{y}=-\frac{mv_{C}\sin\theta}{M+m}
\end{equation}
lấy đạo hàm hai vế phương trình \eqref{eq:311} theo thời gian ta được:
\begin{equation}
  \label{eq:312}
  a_{y}=-\frac{ma_{C}\sin\theta}{M+m}
\end{equation}
giả sử phản lực $N$ hướng lên trên, phương trình chuyển động của khối tâm của hệ trên phương thẳng đứng là:
\begin{equation}
  \label{eq:313}
  (M+m)a_{y}=N-(M+m)g
\end{equation}
từ \eqref{eq:312} và \eqref{eq:313} ta tìm được:
\begin{equation}
  \label{eq:314}
  N=(M+m)g\frac{M+\dfrac{2}{7}m}{M+m-\dfrac{5}{7}m\cos^{2}\theta}
\end{equation}

\noindent\textbf{1d.} Trong hệ quy chiếu gắn với nêm, giả sử phản lực $N_{1}$ do nêm tác dụng lên quả cầu vuông góc với mặt nghiêng của nêm và hướng lên; khi đó, cân bằng lực đối với quả bóng theo phương vuông góc với mặt nghiêng cho:
\begin{equation}
  \label{eq:315}
  N_{1}+ma_{0}\cos\theta=mg\cos\theta
\end{equation}
thay \eqref{eq:39} vào \eqref{eq:315} ta được:
\begin{equation}
  \label{eq:316}
  N_{1}=mg\cos\theta\frac{M+\dfrac{2}{7}m}{M+m-\dfrac{5}{7}m\cos^{2}\theta}
\end{equation}

\noindent\textbf{1e.} Hệ số ma sát tối thiểu là:
\begin{equation}
  \label{eq:317}
  \mu_{0}=\frac{f}{N_{1}}
\end{equation}
thay \eqref{eq:36}, \eqref{eq:310} và \eqref{eq:316} vào \eqref{eq:317} ta được:
\begin{equation}
  \label{eq:318}
  \mu_{0}=\frac{2(M+m)}{(7M+2m)}\tan\theta
\end{equation}

\noindent\textbf{2.} Khi $\mu<\mu_{0}$, quả cầu vừa lăn vừa trượt. Trong hệ quy chiếu gắn với nêm, phương trình chuyển động của quả cầu vẫn được cho bởi \eqref{eq:31}
\begin{equation*}
  ma_{C}=ma_{0}\cos\theta+mg\sin\theta-f
\end{equation*}
lực ma sát khi này không đổi và bằng:
\begin{equation}
  \label{eq:319}
  f=\mu N
\end{equation}
từ \eqref{eq:31}, \eqref{eq:318}, \eqref{eq:315} và \eqref{eq:319} ta tìm được:
\begin{equation}
  \label{eq:320}
  a_{C}=\frac{(M+m)g(\sin\theta-\mu\cos\theta)}{M+m-m\cos^{2}\theta-\mu m\sin\theta\cos\theta}
\end{equation}
\begin{equation}
  \label{eq:321}
  f=\frac{\mu Mmg}{\dfrac{M+m}{\cos\theta}-m\cos\theta-\mu m\sin\theta}
\end{equation}
gia tốc gốc của quả cầu:
\begin{equation}
  \label{eq:322}
  \ddot{\varphi}=\frac{5}{2}\frac{f}{mr}
\end{equation}
vận tốc của điểm tiếp xúc $P$ được tính bởi:
\begin{equation}
  \label{eq:323}
  v_{P}=v_{C}-r\dot{\varphi}=a_{C}t-r\ddot{\varphi}t
\end{equation}
thay \eqref{eq:320}, \eqref{eq:321} và \eqref{eq:322} vào \eqref{eq:323} ta tìm được:
\begin{equation}
  \label{eq:324}
  v_{P}=\frac{(M+m)g\sin\theta-\mu(m+\dfrac{7}{2}M)g\cos\theta}{M+m-m\cos^{2}\theta-\mu m\sin\theta\cos\theta}t
\end{equation}