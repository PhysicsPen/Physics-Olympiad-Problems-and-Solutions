\noindent Tầng thấp nhất của khí quyển là tầng đối lưu, có độ dày khoảng \SI{10}{\kilo\metre}. Trong tầng đối lưu, khi càng lên cao thì nhiệt độ không khí càng giảm. Bên trên tầng đối lưu là tầng bình lưu, trong khoảng \SI{10}{\kilo\metre}, nhiệt độ gần như không thay đổi theo độ cao. Phía trên tầng bình lưu, trong một khoảng độ cao nhất định, nhiệt độ khí quyển lại tăng dần theo độ cao, tầng này được gọi là tầng nghịch (Hình 7).
\begin{enumerate}
  \item Do ánh nắng Mặt Trời, nhiệt độ không khí gần mặt đất cao hơn. Sự thay đổi nhiệt độ khí quyển trong tầng đối lưu có thể xem là kết quả của quá trình biến đổi đoạn nhiệt của không khí. Xác định mối liên hệ giữa nhiệt độ không khí và độ cao tính từ mặt đất. Giả sử nhiệt độ không khí ngay sát mặt đất là $T_{0}$, không khí là khí lý tưởng có khối lượng mol $\mu$, hệ số đoạn nhiệt $\gamma$, và hằng số khí lý tưởng $R$. Gia tốc trọng trường trong tần đối lưu có thể xem là hằng số và bằng $g$.
  \item Vận tốc truyền âm trong không khí dược tính bởi $v_{s}=\sqrt{\left(\dfrac{dp}{d\rho}\right)_{S}}$, trong đó $p$ là áp suất và $\rho$ là khối lượng riêng của không khí. Bỏ qua ảnh hưởng của gió, xác định mối liên hệ giữa vận tốc âm thanh và nhiệt độ $T$ của không khí. Nhiệt độ này thay đổi như thế nào theo độ cao?
  \item Khi sóng âm truyền trong các môi trường có vận tốc âm thanh khác nhau, nó cũng xảy ra hiện tượng phản xạ và khúc xạ, các hiện tượng này xảy ra tương tự với hiện tượng phản xạ và khúc xạ ánh sáng. Để đơn giản hoá, ta có thể coi bề mặt phân cách giữa các lớp khí quyển là phẳng, giả sử nhiệt độ trong tầng đối lưu, tầng nghịch là \SI{-10}{\celsius} và trong tầng bình lưu là \SI{-55}{\celsius}. Trong mô hình này, xét một nguồn âm trong tầng bình lưu, các sóng âm phát ra sẽ đi tới mặt phân cách giữa tầng bình lưu và tầng đối lưu (hay tầng nghịch). Hãy phân tích hiện tượng phản xạ và khúc xạ của sóng âm ứng với các góc tới khác nhau: trong trường hợp nào thì xảy ra khúc xạ?, trong trường hợp nào thì xảy ra phản xạ toàn phần? Đồng thời, vẽ sơ đồ truyền sóng âm cho từng trường hợp. Biết \SI{0}{\celsius}=\SI{273,15}{\kelvin}.
  \item Tiếp tục đơn giản hoá, giả sử nhiệt độ khí quyển không thay đổi theo độ cao, các mặt phân cách là các mặt cầu đồng tâm, nhiệt độ của tầng đối lưu và tầng nghịch giống nhau và đều cao hơn nhiệt độ của tầng bình lưu. Xét một nguồn âm và một khí cầu có gắn một đầu thu âm thanh, cả hai đều có thể ở trong tầng đối lưu hoặc tầng bình lưu. Bỏ quả sự phản xạ sóng âm trên mặt đất cũng như sự hấp thụ năng lượng của sóng âm trong khí quyển. Với 4 trường hợp khác nhau khi nguồn âm và đầu thu nằm lần lượt trong tầng đối lưu và tầng bình lưu, hãy xác định trong trường hợp nào khí cầu có thể phát hiện âm thanh từ khoảng cách xa (vài nghìn \SI{}{\kilo\metre}), và trong trường hợp nào khí cầu chỉ có thể phát hiện âm thanh từ khoảng cách ngắn (dưới \SI{1000}{\kilo\metre}). Giải thích thông qua các tính toán định lượng. Cho biết bán kính Trái Đất và khoảng $R_{e}=$\SI{6371}{\kilo\metre}.
\end{enumerate}