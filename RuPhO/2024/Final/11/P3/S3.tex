\noindent\textbf{1.} Xét một diện tích nguyên tố $dS$ của mặt đáy, gọi $\vec{e}_{n}$ là vector đơn vị theo hướng pháp tuyến của mặt đáy và $\vec{r}$ là vector khoảng cách từ diện tích nguyên tố đang xét đến điểm $O$. Khi đó, thành phần pháp tuyến của điện trường do diện tích này gây ra tại $O$ là:
\begin{equation*}
  dE_{n}=\left(\frac{\sigma\vec{r}dS}{4\pi\varepsilon_{0}r^{3}}\right)\cdot\vec{e}_{n}=\frac{\sigma}{4\pi\varepsilon_{0}}\frac{\vec{r}\cdot d\vec{S}}{r^{3}}=\frac{\sigma d\Omega_{A}}{4\pi\varepsilon_{0}}\implies E_{n}=\frac{\sigma \Omega_{A}}{4\pi\varepsilon_{0}}
\end{equation*}
vì cả hai đáy đều được nhìn từ điểm $O$ dưới góc khối như nhau và $\sigma_{1}=-\sigma_{2}$, điện trường tổng hợp do hai đáy gây ra tại $O$ bằng không:
\begin{equation*}
  E_{O}^{\text{bases}}=0
\end{equation*}
chia mặt bên thành các vòng dây nhỏ, vòng dây ứng với khoảng cách $x$ sẽ có bán kính $r=x\sin\alpha$ và điện tích $dq=\sigma(x)\cdot 2\pi rdx$, như vậy, điện trường do mặt bên gây ra tại $O$ là:
\begin{equation*}
  \vec{E}_{0}^{\text{side}}=\int\frac{1}{4\pi\varepsilon_{0}\frac{\cos\alpha dq}{x^{2}}}=\int_{l}^{2l}\frac{1}{4\pi\varepsilon_{0}}\frac{2\pi x\sin\alpha}{x^{2}}\frac{A}{x}\cos\alpha dx=\frac{\sqrt{3}}{16}\frac{A}{\varepsilon_{0}l}
\end{equation*}
như vậy, cường độ điện trường tại $O$ là:
\begin{equation*}
  E_{O}=E_{O}^{\text{bases}}+E_{O}^{\text{side}}=\frac{\sqrt{3}}{16}\frac{A}{\varepsilon_{0}l}
\end{equation*}

\noindent\textbf{2.} Gọi $z$ là khoảng cách từ điện tích thử đến điểm $O$. Cường độ điện trường tại $z$ do mặt bên của hình nón cụt gây ra là:
\begin{equation*}
  E^{\text{side}}(z)=\int_{l}^{2l}=\frac{1}{4\pi\varepsilon_{0}}\frac{2\pi x\sin\alpha A(x\cos\alpha-z)}{[(x\sin\alpha)^{2}+(x\cos\alpha-z)^{2}]^{3/2}}dx=\frac{A}{4\varepsilon_{0}}\int_{l}^{2l}\frac{x\cos\alpha-z}{[(x\sin\alpha)^{2}+(x\cos\alpha-z)^{2}]^{3/2}}
\end{equation*}
suy ra:
\begin{equation*}
  E^{\text{side}}(z)=\frac{l}{z}\left(\frac{2\sqrt{l^{2}-\sqrt{3}lz+z^{2}}-\sqrt{4l^{2}-2\sqrt{3}lz+z^{2}}}{\sqrt{l^{2}-\sqrt{3}lz+z^{2}}\sqrt{4l^{2}-2\sqrt{3}lz+z^{2}}}\right)
\end{equation*}
điện trường này bằng không tại:
\begin{equation*}
  z_{0}=\frac{2}{\sqrt{3}}l
\end{equation*}

\noindent Như ta đã biết, điện trường do một đĩa phẳng bán kính $R$, tích điện đều với mật độ điện tích mặt $\sigma$ gây ra tại một điểm nằm trên trục đối xứng, các tâm đĩa một khoảng $y$ là:
\begin{equation*}
  E^{\text{disk}}(y)=\frac{\sigma}{2\varepsilon_{0}}\left(1-\frac{y}{\sqrt{R^{2}+y^{2}}}\right)
\end{equation*}
đối với đáy trên:
\begin{equation*}
  R_{1}=\frac{l}{2}\quad;\quad y_{1}=z-\frac{\sqrt{3}}{2}l
\end{equation*}
và đối với đáy dưới:
\begin{equation*}
  R_{2}=l\quad;\quad y_{2}=\sqrt{3}l-z
\end{equation*}
như vậy, điện trường do hai đáy gây ra, tại điểm các $O$ một khoảng $z$ là:
\begin{equation*}
  E^{\text{bases}}(z)=\frac{\sigma_{0}}{2\varepsilon_{0}}\left(\frac{\sqrt{3}l-z}{\sqrt{4l^{2}-2\sqrt{3}lz+z^{2}}}-\frac{z-\dfrac{\sqrt{3}}{2}l}{\sqrt{l^{2}-\sqrt{3}lz+z^{2}}}\right)
\end{equation*}
điện trường này bằng không tại:
\begin{equation*}
  z_{0}'=\frac{2}{\sqrt{3}}l
\end{equation*}
vì $z_{0}=z_{0}'$ nên điện tích thử cân bằng tại vị trí nằm cách $O$ một khoảng $\dfrac{2}{\sqrt{e}}l$. Khoảng cách này không phụ thuộc vào $\sigma_{0}$.\\

\noindent\textbf{3.} Đặt $z=z_{0}+\delta z$ với $\delta z\gg l$, khi đó ta có thể viết:
\begin{equation*}
  E^{\text{side}}(\delta z)\approx\frac{9\sqrt{3}}{32}\frac{A}{\varepsilon_{0}l^{2}}\delta z
\end{equation*}
và:
\begin{equation*}
  E^{\text{bases}}(\delta z)\approx-\frac{9\sqrt{3}}{16}\frac{\sigma_{0}}{\varepsilon_{0}l}\delta z
\end{equation*}
phương trình chuyển động của điện tích thử lúc này:
\begin{equation*}
  \delta\ddot{z}+\frac{9\sqrt{3}}{32}\frac{q}{m\varepsilon_{0}l}\left(\frac{A}{l}-2\sigma_{0}\right)\delta z=0
\end{equation*}
để vị trí cân bằng tại $z=z_{0}$ là bền, điện tích thử phải dao động bé quanh vị trí này, điều này đòi hỏi:
\begin{equation*}
  \sigma_{0}<\frac{A}{2l}
\end{equation*}
chu kỳ dao động bé:
\begin{equation*}
  T=\frac{8\pi}{3}\sqrt{\frac{2}{\sqrt{3}}\frac{m\varepsilon_{0}l}{q}\left(\frac{A}{l}-2\sigma_{0}\right)^{-1}}
\end{equation*}