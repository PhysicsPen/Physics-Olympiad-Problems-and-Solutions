\noindent Môt lượng nhỏ thuỷ ngân được đổ lên mặt trên của một hình trụ đặc có tiết diện $S$. Biết góc thấm ướt của vật liệu làm nên hình trụ là $\theta$, hệ số căng mặt ngoài của thuỷ ngân là $\sigma$, khối lượng riêng của thuỷ ngân là $\rho$ và gia tốc trọng trường có độ lớn $g$. Giả sử các thông số đã cho thoả mãn bất đẳng thức:
\begin{equation*}
  \sqrt{\frac{\sigma}{\rho g}}\ll S
\end{equation*}
\begin{enumerate}
  \item Xác định thể tích tối thiểu cần đổ $V_{0}$ để thuỷ ngân có thể bao phủ toàn bộ đáy trên của hình trụ.
\end{enumerate}
\noindent Một lượng thuỷ ngân có thể tích $V<V_{0}$ được đổ lên mặt trên của hình trụ. Sau đó, một hình trụ khác có cùng tiết diện nhưng được làm bằng vật liệu không dính ướt thuỷ ngân được đặt cẩn thận lên trên. Giả sử các hình trụ và thuỷ ngân luôn có tính đối xứng trục.
\begin{enumerate}
  \setcounter{enumi}{1}
  \item Khối lượng tối thiểu $m_{1}$ của hình trụ phía trên là bao nhiêu để thuỷ ngân có thể tiếp xúc hoàn toàn với đáy trên của khối trụ phía dưới?
  \item Khối lượng tối thiểu $m_{2}$ của hình trụ phía trên là bao nhiêu để thuỷ ngân có thể chảy ra từ khe hở giữa hai khối trụ? Giả sử đường biên giữa đáy dưới của hình trụ và mặt bên của nó là đường tròn có bán kính rất nhỏ so với độ dày của lớp thuỷ ngân.
\end{enumerate}
